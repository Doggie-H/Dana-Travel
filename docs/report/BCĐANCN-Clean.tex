% ============================================================
% BÁAO CÁO ĐỒ ÁN CHUYÊN NGÀNH - DanangTravel-AI
% ============================================================
% Tuân thủ quy tắc trình bày theo tiêu chuẩn TCVN 6909:2001
% ============================================================

\documentclass[14pt,a4paper]{extarticle}

% ============================================================
% 1. LAYOUT & GEOMETRY
% ============================================================
\usepackage[a4paper, top=20mm, bottom=20mm, left=30mm, right=20mm]{geometry}

% ============================================================
% 2. FONTS (LuaLaTeX)
% ============================================================
\usepackage{fontspec}
\setmainfont{Times New Roman}
\usepackage{polyglossia}
\setdefaultlanguage{vietnamese}

% ============================================================
% 3. SPACING & INDENTATION
% ============================================================
\usepackage{parskip}
\setlength{\parindent}{1.27cm}  % First line indent 1.27cm
\setlength{\parskip}{6pt plus 2pt minus 1pt}  % Paragraph spacing 6pt
\usepackage{setspace}
\setstretch{1.4}  % Line spacing 1.4 (between 1.3-1.5)

% ============================================================
% 4. TEXT ALIGNMENT & JUSTIFICATION
% ============================================================
\usepackage{ragged2e}
\justifying

% ============================================================
% 5. SECTION FORMATTING
% ============================================================
\usepackage{titlesec}
\renewcommand{\thesection}{\Roman{section}}
\renewcommand{\thesubsection}{\arabic{section}.\arabic{subsection}}
\renewcommand{\thesubsubsection}{\thesubsection.\arabic{subsubsection}}

\titleformat{\section}
  {\normalfont\bfseries\large}
  {\thesection.}
  {1em}
  {}

\titleformat{\subsection}
  {\normalfont\bfseries}
  {\thesubsection.}
  {1em}
  {}

\titleformat{\subsubsection}
  {\normalfont\bfseries}
  {\thesubsubsection.}
  {1em}
  {}

% ============================================================
% 6. HEADERS & FOOTERS
% ============================================================
\usepackage{fancyhdr}
\pagestyle{fancy}
\fancyhf{}
\fancyfoot[C]{\thepage}
\renewcommand{\headrulewidth}{0pt}

% ============================================================
% 7. TABLE OF CONTENTS
% ============================================================
\usepackage{titletoc}

% ============================================================
% 8. GRAPHICS & COLORS
% ============================================================
\usepackage{graphicx}
\usepackage{xcolor}
\usepackage{eso-pic}
\usepackage{tikz}

% ============================================================
% 9. BOXES & COLORED TEXT
% ============================================================
\usepackage[most]{tcolorbox}
\usepackage{mdframed}

% ============================================================
% 10. LISTS & TABLES
% ============================================================
\usepackage{array}
\usepackage{longtable}
\usepackage{multirow}
\usepackage{booktabs}
\usepackage{listings}

% ============================================================
% 11. CODE LISTING STYLE
% ============================================================
\lstdefinestyle{pythonstyle}{
  language=Python,
  basicstyle=\ttfamily\footnotesize,
  keywordstyle=\color{blue}\bfseries,
  commentstyle=\color{gray}\itshape,
  stringstyle=\color{red},
  numbers=left,
  numberstyle=\tiny\color{gray},
  stepnumber=1,
  numbersep=8pt,
  tabsize=4,
  breaklines=true,
  showspaces=false,
  showstringspaces=false,
  frame=single,
  rulecolor=\color{black},
  backgroundcolor=\color{white},
  captionpos=b,
  keepspaces=true,
}

\renewcommand{\lstlistingname}{Mô tả}

\newtcblisting{pythoncode}[2][]{
  listing engine=listings,
  colback=white,
  colframe=black,
  listing only,
  listing style=pythonstyle,
  left=5pt,
  right=5pt,
  top=5pt,
  bottom=5pt,
  enhanced,
  sharp corners,
  boxrule=0.5pt,
  title=#2,
  fonttitle=\bfseries,
  #1
}

% ============================================================
% 12. MISCELLANEOUS
% ============================================================
\usepackage{amsmath}
\usepackage{amssymb}
\usepackage{pifont}
\usepackage{fmtcount}
\usepackage{hyperref}

% ============================================================
% BEGIN DOCUMENT
% ============================================================

\begin{document}

% ============================================================
% COVER PAGE (Bìa - Trang 1)
% ============================================================
\pagestyle{empty}
\pagenumbering{gobble}

\begin{center}
  \vspace*{1.5cm}
  \Large\textbf{ĐẠI HỌC ĐÀ NẴNG}\\[0.3em]
  \Large\textbf{TRƯỜNG ĐẠI HỌC SƯ PHẠM}\\[1.5em]
  \large\textbf{KHOA TOÁN - TIN}
\end{center}

\vspace{3cm}

\begin{center}
  \LARGE\textbf{ĐỀ CƯƠNG ĐỒ ÁN CHUYÊN NGÀNH}\\[1em]
  \Large\textbf{ĐỀ TÀI}\\[0.5em]
  \Large\textbf{PHÁT TRIỂN WEBSITE CHATBOT TƯ VẤN LỊCH TRÌNH DU LỊCH ĐÀ NẴNG}\\[0.5em]
\end{center}

\vfill

\begin{center}
  \begin{tcolorbox}[
    colback=gray!5,
    colframe=black,
    coltitle=black,
    boxrule=2pt,
    arc=5pt,
    width=0.8\textwidth,
    left=10pt,
    right=10pt,
    top=10pt,
    bottom=10pt,
    enhanced,
  ]
    \centering
    \large
    \textbf{Giảng viên hướng dẫn:} TS. Trần Văn Hưng\\[0.5em]
    \textbf{Ngành:} Công Nghệ Thông Tin\\[0.5em]
    \textbf{Lớp sinh hoạt:} 23CNTT3\\[0.5em]
    \textbf{Sinh viên thực hiện:} Huỳnh Kim Đô\\[0.5em]
    \textbf{MSSV:} 3120223031
  \end{tcolorbox}
\end{center}

\vspace{1.5cm}

\begin{center}
  \large\textbf{Đà Nẵng, tháng 12 năm 2025}
\end{center}

% ============================================================
% PAGE 2: YÝ KIẾN CỦA GIẢNG VIÊN HƯỚNG DẪN
% ============================================================
\newpage
\pagestyle{empty}

\begin{center}
  \Large\textbf{Ý KIẾN CỦA GIẢNG VIÊN HƯỚNG DẪN}
\end{center}

\vspace{2cm}

\noindent
\begin{tcolorbox}[
  colback=white,
  colframe=gray,
  boxrule=1pt,
  left=10pt,
  right=10pt,
  top=10pt,
  bottom=10pt,
]
  \vspace{8cm}
  
  \noindent Đà Nẵng, ngày \hspace{2cm} tháng \hspace{2cm} năm 2025
  
  \vspace{2cm}
  
  \begin{flushright}
    \textbf{Giảng viên hướng dẫn}\\[3cm]
    (Ký tên và ghi họ tên)
  \end{flushright}
\end{tcolorbox}

% ============================================================
% PAGE 3: NHẬN XÉT CỦA GIẢNG VIÊN CHẤM ĐỒ ÁN
% ============================================================
\newpage

\begin{center}
  \Large\textbf{NHẬN XÉT CỦA GIẢNG VIÊN CHẤM ĐỒ ÁN}
\end{center}

\vspace{2cm}

\noindent
\begin{tcolorbox}[
  colback=white,
  colframe=gray,
  boxrule=1pt,
  left=10pt,
  right=10pt,
  top=10pt,
  bottom=10pt,
]
  \vspace{8cm}
  
  \noindent Đà Nẵng, ngày \hspace{2cm} tháng \hspace{2cm} năm 2025
  
  \vspace{2cm}
  
  \begin{flushright}
    \textbf{Giảng viên chấm đồ án}\\[3cm]
    (Ký tên và ghi họ tên)
  \end{flushright}
\end{tcolorbox}

% ============================================================
% TABLE OF CONTENTS
% ============================================================
\newpage
\pagestyle{fancy}
\fancyhf{}
\fancyfoot[C]{\thepage}
\pagenumbering{arabic}
\setcounter{page}{1}

\tableofcontents

% ============================================================
% CONTENT SECTIONS (4-6 pages total)
% ============================================================

\newpage
\section{ĐẶT VẤN ĐỀ}
\label{sec:intro}

\noindent
DanangTravel-AI là ứng dụng web full-stack được phát triển nhằm tự động hoá quá trình lập kế hoạch du lịch tại Đà Nẵng. Hệ thống kết hợp thuật toán tối ưu hoá đa pha, công nghệ trí tuệ nhân tạo (AI), và giao diện web hiện đại để giải quyết một vấn đề thực tế: du khách mất nhiều thời gian tìm kiếm thông tin từ nhiều nguồn khác nhau khi chuẩn bị cho một chuyến du lịch.

\subsection{Nền tảng và bối cảnh vấn đề}

Hiện nay, khi du khách muốn lên kế hoạch cho một chuyến du lịch tại Đà Nẵng, họ phải tự mình:

\begin{itemize}
  \item Tìm kiếm thông tin từ Google Maps, blog du lịch, diễn đàn
  \item Ghi chú lại giá vé, giờ mở cửa, vị trí địa điểm
  \item Tính toán chi phí ăn ở, vé vào, di chuyển
  \item Sắp xếp thứ tự địa điểm hợp lý
  \item Điều chỉnh lịch trình khi phát hiện vấn đề về thời gian hoặc chi phí
\end{itemize}

Quy trình này tốn thời gian, dễ xảy ra sai sót, và không đảm bảo lịch trình được tối ưu về chi phí hay thời gian di chuyển. Ngoài ra, khi có thay đổi nhu cầu, việc điều chỉnh lịch trình từ đầu là rất phức tạp.

\subsection{Vấn đề cần giải quyết}

Chính vì những khó khăn trên, hệ thống DanangTravel-AI được thiết kế để:

\begin{enumerate}
  \item \textbf{Tự động tạo lịch trình:} Chỉ cần nhập 6 thông tin cơ bản (ngân sách, số người, ngày đến/về, loại lưu trú, phương tiện, sở thích), hệ thống sẽ sinh ra lịch trình chi tiết cho từng ngày.
  
  \item \textbf{Tối ưu hoá chi phí:} Phân bổ ngân sách hợp lý giữa lưu trú, ăn uống, vé tham quan, và di chuyển; đảm bảo không vượt quá tổng ngân sách được cấp.
  
  \item \textbf{Giảm thời gian di chuyển:} Sử dụng Google Maps API để tính khoảng cách thực tế giữa các địa điểm, từ đó sắp xếp thứ tự tham quan sao cho tổng thời gian di chuyển là nhỏ nhất.
  
  \item \textbf{Hỗ trợ bằng AI:} Cung cấp trợ lý chatbot tương tác để trả lời câu hỏi, gợi ý, thay đổi lịch trình theo yêu cầu của người dùng.
  
  \item \textbf{Quản lý dữ liệu:} Cấp quyền quản trị viên để thêm, sửa, xóa thông tin về 120+ địa điểm, theo dõi xu hướng tìm kiếm, ghi nhận lỗi hệ thống.
\end{enumerate}

\subsection{Phạm vi và mục tiêu của đồ án}

Đồ án này tập trung vào:

\begin{itemize}
  \item \textbf{Phạm vi chức năng:} Xây dựng một hệ thống web hoàn chỉnh với ba phần: tạo lịch trình tự động (core feature), chatbot AI tư vấn, bảng điều khiển quản trị.
  
  \item \textbf{Phạm vi công nghệ:} Sử dụng React 18 (Frontend), Node.js/Express (Backend), Prisma ORM, SQLite (dev) / PostgreSQL (prod), Google Gemini API, Google Maps API.
  
  \item \textbf{Mục tiêu kỹ thuật:}
  \begin{itemize}
    \item Thiết kế kiến trúc Three-Tier rõ ràng, dễ bảo trì và mở rộng
    \item Triển khai thuật toán lập lịch 7 pha với kiểm tra ràng buộc (giờ mở cửa, chi phí, khoảng cách)
    \item Tích hợp AI một cách an toàn bằng adapter pattern
    \item Ghi nhận các chỉ số quan trọng: lượt truy cập, xu hướng tìm kiếm, lỗi hệ thống
  \end{itemize}
\end{itemize}

Bằng cách triển khai hệ thống này, du khách có thể lập kế hoạch cho chuyến đi của mình một cách nhanh chóng, chính xác, và tối ưu.

\subsection{Cấu trúc báo cáo}

Báo cáo này được tổ chức theo cấu trúc sau:

\begin{itemize}
  \item \textbf{Phần I (đang đọc):} Giới thiệu vấn đề, bối cảnh, phạm vi và mục tiêu đồ án.
  \item \textbf{Phần II:} Trình bày kiến trúc tổng thể, cách thức phân tầng, công nghệ sử dụng.
  \item \textbf{Phần III:} Mô tả chi tiết thiết kế cơ sở dữ liệu, sơ đồ ERD, cấu trúc API, các thành phần chính.
  \item \textbf{Phần IV:} Giải thích chi tiết thuật toán tạo lịch trình, chatbot AI, quản trị hệ thống, và đánh giá hiệu suất.
  \item \textbf{Phần V:} Kết luận, nhận xét về ưu/nhược điểm, hướng phát triển tương lai.
  \item \textbf{Tài liệu tham khảo:} Danh sách các nguồn tài liệu, API, framework được sử dụng.
  \item \textbf{Phụ lục:} Bộ dữ liệu mẫu, code snippet, sơ đồ chi tiết.
\end{itemize}

\newpage
\section{PHƯƠNG PHÁP VÀ KIẾN TRÚC}
\label{sec:method}

\noindent
Phần này trình bày phương pháp, thiết kế kiến trúc, công nghệ sử dụng, và cách tiếp cận chính.

\subsection{Kiến trúc hệ thống}

Nội dung...

\subsection{Công nghệ sử dụng}

Nội dung...

\newpage
\section{THIẾT KẾ CHI TIẾT}
\label{sec:design}

\noindent
Phần này mô tả thiết kế chi tiết các thành phần, module, database, API, giao diện, v.v.

\subsection{Sơ đồ ERD}

Nội dung...

\subsection{Luồng dữ liệu}

Nội dung...

\newpage
\section{KẾT QUẢ VÀ ĐÁNH GIÁ}
\label{sec:results}

\noindent
Phần này trình bày kết quả thực hiện, đánh giá hiệu suất, ưu/nhược điểm, và hướng phát triển.

\subsection{Kết quả đạt được}

Nội dung...

\subsection{Đánh giá}

Nội dung...

% ============================================================
% CONCLUSION
% ============================================================

\newpage
\section{KẾT LUẬN}
\label{sec:conclusion}

\noindent
Phần kết luận tóm tắt những điểm chính, đóng góp của đồ án, và hướng phát triển tương lai.

% ============================================================
% REFERENCES
% ============================================================

\newpage
\section{TÀI LIỆU THAM KHẢO}
\label{sec:references}

\begin{enumerate}
  \item Trần Văn Hưng (2024). Giáo trình phát triển ứng dụng web.
  \item Tài liệu React 18. https://react.dev
  \item Tài liệu Express.js. https://expressjs.com
  \item Tài liệu Prisma ORM. https://www.prisma.io
\end{enumerate}

% ============================================================
% APPENDIX
% ============================================================

\newpage
\section{PHỤ LỤC}
\label{sec:appendix}

\subsection{PHỤ LỤC A: Bộ dữ liệu}

Nội dung...

\subsection{PHỤ LỤC B: Code mẫu}

\begin{lstlisting}[language=Python, caption=Ví dụ code]
# Code mẫu
def hello():
    print("Hello, World!")
\end{lstlisting}

% ============================================================
% END DOCUMENT
% ============================================================

\end{document}
