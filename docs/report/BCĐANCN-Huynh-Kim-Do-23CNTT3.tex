% Options for packages loaded elsewhere
\PassOptionsToPackage{unicode}{hyperref}
\PassOptionsToPackage{hyphens}{url}
\documentclass[14pt,a4paper]{extarticle}
\usepackage{xcolor}
\usepackage{amsmath,amssymb}
% --- Added for layout and cover styling ---
\usepackage[a4paper, top=20mm, bottom=20mm, left=30mm, right=20mm]{geometry}
\usepackage{titlesec}
\usepackage{fmtcount}
\renewcommand{\thesection}{\Roman{section}}
\renewcommand{\thesubsection}{\arabic{section}.\arabic{subsection}}
\renewcommand{\thesubsubsection}{\thesubsection.\arabic{subsubsection}}
\titleformat{\section}{\normalfont\bfseries\large}{\thesection.}{1em}{}
\titleformat{\subsection}{\normalfont\bfseries}{\thesubsection.}{1em}{}
\titleformat{\subsubsection}{\normalfont\bfseries}{\thesubsubsection.}{1em}{}
\usepackage[most]{tcolorbox}
\usepackage{titletoc}
\usepackage{ragged2e}
\justifying
\usepackage{setspace}
\setstretch{1.4}
\usepackage{fancyhdr}
\pagestyle{fancy}
\fancyhf{}
\fancyfoot[C]{\thepage}
\renewcommand{\headrulewidth}{0pt}
\usepackage{listings}
\renewcommand{\lstlistingname}{Mô tả}
\lstdefinestyle{pythonstyle}{
  language=Python,
  basicstyle=\ttfamily\footnotesize,
  keywordstyle=\color{blue}\bfseries,
  commentstyle=\color{gray}\itshape,
  stringstyle=\color{red},
  numbers=left,
  numberstyle=\tiny\color{gray},
  stepnumber=1,
  numbersep=8pt,
  tabsize=4,
  breaklines=true,
  showspaces=false,
  showstringspaces=false,
  frame=single,
  rulecolor=\color{black},
  backgroundcolor=\color{white},
  captionpos=b,
  keepspaces=true,
}
\newtcblisting{pythoncode}[2][]{
  listing engine=listings,
  colback=white,
  colframe=black,
  listing only,
  listing style=pythonstyle,
  left=5pt,
  right=5pt,
  top=5pt,
  bottom=5pt,
  enhanced,
  sharp corners,
  boxrule=0.5pt,
  title=#2,
  fonttitle=\bfseries,
  #1
}
\usepackage{mdframed}
\usepackage{eso-pic}
\usepackage{tikz}
\usepackage{pifont}
\usepackage{array}
\usepackage{longtable}
\usepackage{multirow}
\usepackage{booktabs}
% LuaLaTeX font setup
\usepackage{fontspec}
\setmainfont{Times New Roman}
\usepackage{polyglossia}
\setdefaultlanguage{vietnamese}

\begin{document}
  % --- COVER PAGE (inserted) ---
  \pagestyle{empty}
  \AddToShipoutPictureBG*{%
    \put(-205,-120){%
      \includegraphics[width=35.2cm,height=37.75cm]{D:/Study/tailieu/khungbia.png}%
    }%
  }
  \pagenumbering{gobble}

  \begin{center}
    \Large\textbf{ĐẠI HỌC ĐÀ NẴNG}\\[0.5em]
    \Large\textbf{TRƯỜNG ĐẠI HỌC SƯ PHẠM}\\[2em]
    \begin{tikzpicture}[remember picture,overlay]
      \node at (2.32cm,1.5cm) {\includegraphics[width=20.5cm]{D:/Study/tailieu/duongphanchia.png}};
    \end{tikzpicture}
    \large\textbf{KHOA TOÁN TIN}
  \end{center}
  \begin{tikzpicture}[remember picture,overlay]
    \node at (7.8cm,-2cm) {\includegraphics[width=5.5cm]{D:/Study/tailieu/icon.png}};
  \end{tikzpicture}
  \vspace{4cm}

  \begin{center}
    \LARGE\textbf{ĐỀ CƯƠNG ĐỒ ÁN CHUYÊN NGÀNH}\\[1em]
    \Large\textbf{ĐỀ TÀI}\\[0.5em]
    \Large\textbf{PHÁT TRIỂN WEBSITE BÁN HÀNG BOARD GAME VỚI CHATBOT TƯ VẤN BÁN HÀNG}\\[0.5em]
  \end{center}
  \vfill
  \begin{center}
    \begin{tcolorbox}[
      colback=gray!5,
      colframe=black,
      coltitle=black,
      boxrule=3.5pt,
      arc=10pt,
      width=0.8\textwidth,
      height=5cm,
      drop shadow=black!30,
      enhanced,
    ]
      \begin{spacing}{1.5}
        \large
        \textbf{Giảng viên hướng dẫn:} TS.Trần Văn Hưng\\
        \textbf{Ngành:} Công Nghệ Thông Tin\\
        \textbf{Lớp sinh hoạt:} 23CNTT3\\
        \textbf{Sinh viên thực hiện:} Huỳnh Kim Đô
      \end{spacing}
    \end{tcolorbox}
  \end{center}
  \vspace{0.5cm}

  \begin{center}
    \large\textbf{Đà Nẵng, năm 2025}
  \end{center}

  \newpage

  % --- TABLE OF CONTENTS ---
  \tableofcontents
  \newpage

  % --- PAGE STYLE AFTER COVER ---
  \pagestyle{fancy}
  \fancyhf{}
  \fancyfoot[C]{\thepage}
  \renewcommand{\headrulewidth}{0pt}
  \pagenumbering{arabic}
  \setcounter{page}{1}

  % --- START CONTENT ---
  \newpage
  \section{ĐẶT VẤN ĐỀ}
  \label{sec:intro}

% Content for ĐẶT VẤN ĐỀ section goes here

  \newpage

\textbf{................................................................................................................}

\textbf{................................................................................................................}

\textbf{................................................................................................................}

\textbf{................................................................................................................}

\textbf{................................................................................................................}

\textbf{................................................................................................................}

\textbf{................................................................................................................}

\textbf{................................................................................................................}

\textbf{................................................................................................................}

\textbf{................................................................................................................}

\textbf{................................................................................................................}

\textbf{................................................................................................................}

\textbf{................................................................................................................}

\textbf{................................................................................................................}

\textbf{................................................................................................................}

\textbf{................................................................................................................}

\textbf{................................................................................................................}

\begin{quote}
Đà Nẵng, ngày \ldots\ldots. tháng ...\ldots.. năm 2025
\end{quote}

\textbf{Giảng viên hướng dẫn}

\textbf{\hfill\break
}

\textbf{NHẬN XÉT CỦA GIẢNG VIÊN CHẤM ĐỒ ÁN}

\textbf{................................................................................................................}

\textbf{................................................................................................................}

\textbf{................................................................................................................}

\textbf{................................................................................................................}

\textbf{................................................................................................................}

\textbf{................................................................................................................}

\textbf{................................................................................................................}

\textbf{................................................................................................................}

\textbf{................................................................................................................}

\textbf{................................................................................................................}

\textbf{................................................................................................................}

\textbf{................................................................................................................}

\textbf{................................................................................................................}

\textbf{................................................................................................................}

\textbf{................................................................................................................}

\textbf{................................................................................................................}

\textbf{................................................................................................................}

\begin{quote}
Đà Nẵng, ngày \ldots\ldots.. tháng \ldots\ldots. năm 2025

\textbf{Giảng viên chấm đồ án}
\end{quote}

\textbf{LỜI CẢM ƠN}

Lời đầu tiên, em xin gửi lời cảm ơn chân thành đến các Thầy, Cô giáo
trong Khoa Tin học -- Trường Đại học Sư phạm, Đại học Đà Nẵng đã tận
tình giảng dạy và trang bị cho em những kiến thức nền tảng quý báu trong
suốt thời gian học tập tại trường.

Đặc biệt, em xin bày tỏ lòng biết ơn sâu sắc đến thầy PGS.TS Trần Văn
Hưng Thầy đã dành thời gian quý báu để định hướng đề tài, sửa chữa những
sai sót và chỉ dẫn tận tình giúp em hoàn thành đồ án này đúng tiến độ.

Đồ án "Xây dựng ứng dụng lập lịch trình và hỗ trợ du lịch Đà Nẵng
(DanaTravel)" là kết quả của quá trình nỗ lực học hỏi và tìm tòi công
nghệ mới (ReactJS, Node.js, AI Gemini). Tuy nhiên, do kiến thức và kinh
nghiệm thực tế của em còn hạn chế, nên báo cáo khó tránh khỏi những
thiếu sót. Em rất mong nhận được sự đóng góp ý kiến của quý Thầy, Cô để
đề tài được hoàn thiện hơn, cũng như để em rút kinh nghiệm cho các dự án
sau này

Em xin chân thành cảm ơn!

\begin{quote}
\textbf{Huỳnh Kim Đô}
\end{quote}

\textbf{\hfill\break
}

\newpage
\section{Mục Lục}\label{mux1ee5c-lux1ee5c}

\hyperref[mux1edf-ux111ux1ea7u]{MỞ ĐẦU
\hyperref[mux1edf-ux111ux1ea7u]{8}}

\hyperref[chux1b0ux1a1ng-1-cux1a1-sux1edf-luxfd-thuyux1ebft]{CHƯƠNG 1:
CƠ SỞ LÝ THUYẾT
\hyperref[chux1b0ux1a1ng-1-cux1a1-sux1edf-luxfd-thuyux1ebft]{10}}

\hyperref[_Toc215923986]{1.1 Các chuẩn web nền tảng (Web Standards)
\hyperref[_Toc215923986]{10}}

\hyperref[_Toc215923987]{1.1.1 Tổng quan về HTML 5
\hyperref[_Toc215923987]{10}}

\hyperref[_Toc215923988]{1.1.2 Tổng quan về CSS
\hyperref[_Toc215923988]{10}}

\hyperref[_Toc215923989]{1.2 Công nghệ phát triển Giao diện (Frontend)
\hyperref[_Toc215923989]{11}}

\hyperref[_Toc215923990]{1.2.1 Ngôn ngữ lập trình JavaScript (ES6+)
\hyperref[_Toc215923990]{11}}

\hyperref[_Toc215923991]{1.2.2 Tổng quan về ReactJS
\hyperref[_Toc215923991]{11}}

\hyperref[_Toc215923992]{1.2.3 Framework Tailwind CSS
\hyperref[_Toc215923992]{12}}

\hyperref[_Toc215923993]{1.3 Công nghệ phát triển Hệ thống máy chủ
(Backend) \hyperref[_Toc215923993]{13}}

\hyperref[_Toc215923994]{1.3.1 Tổng quan về Node.js
\hyperref[_Toc215923994]{13}}

\hyperref[_Toc215923995]{1.3.2 Framework Express.js
\hyperref[_Toc215923995]{14}}

\hyperref[_Toc215923996]{1.3.3 Chuẩn RESTful API
\hyperref[_Toc215923996]{14}}

\hyperref[_Toc215923997]{1.4 Hệ quản trị Cơ sở dữ liệu (Database)
\hyperref[_Toc215923997]{15}}

\hyperref[_Toc215923998]{1.4.1 SQLite \hyperref[_Toc215923998]{15}}

\hyperref[_Toc215923999]{1.4.2 Prisma ORM (Object-Relational Mapping)
\hyperref[_Toc215923999]{16}}

\hyperref[_Toc215924000]{1.5 Trí tuệ nhân tạo và Mô hình ngôn ngữ lớn
(LLM) \hyperref[_Toc215924000]{17}}

\hyperref[_Toc215924001]{1.5.1 Mô hình Generative AI (Google Gemini)
\hyperref[_Toc215924001]{17}}

\hyperref[_Toc215924002]{1.5.2 Kỹ thuật RAG (Retrieval-Augêmnted
Generation) \hyperref[_Toc215924002]{18}}

\hyperref[_Toc215924003]{1.6 Lý thuyết bài toán xếp lịch (Scheduling)
\hyperref[_Toc215924003]{18}}

\hyperref[_Toc215924004]{1.6.1 Bài toán thỏa mãn ràng buộc (CSP)
\hyperref[_Toc215924004]{18}}

\hyperref[_Toc215924005]{1.6.2 Giải thuật Heuristic (Heuristic
Algorithm) \hyperref[_Toc215924005]{19}}

\hyperref[chux1b0ux1a1ng-2-phuxe2n-tuxedch-vuxe0-thiux1ebft-kux1ebf-web]{CHƯƠNG
2: PHÂN TÍCH VÀ THIẾT KẾ WEB
\hyperref[chux1b0ux1a1ng-2-phuxe2n-tuxedch-vuxe0-thiux1ebft-kux1ebf-web]{21}}

\hyperref[_Toc215924007]{2.1 Phân tích Tác nhân hệ thống (System Actors
Analysis) \hyperref[_Toc215924007]{21}}

\hyperref[_Toc215924008]{2.1.1 Tác nhân: Khách du lịch vãng lai (Guest
User) \hyperref[_Toc215924008]{21}}

\hyperref[_Toc215924009]{2.1.2 Tác nhân: Quản trị viên hệ thống (System
Administrator) \hyperref[_Toc215924009]{21}}

\hyperref[_Toc215924010]{2.2 Xây dựng Chân dung người dùng (User
Personas \& Algorithm Mapping) \hyperref[_Toc215924010]{22}}

\hyperref[_Toc215924011]{2.2.1 Persona A: "Sinh viên du lịch bụi" (The
Budget Backpacker) \hyperref[_Toc215924011]{22}}

\hyperref[_Toc215924012]{2.2.2 Persona B: "Gia đình nghỉ dưỡng" (The
Family Vacationer) \hyperref[_Toc215924012]{22}}

\hyperref[_Toc215924013]{2.2.3 Persona C: "Cặp đôi lãng mạn" (The
Honeymooners) \hyperref[_Toc215924013]{23}}

\hyperref[_Toc215924014]{2.3 Đặc tả chi tiết yêu cầu chức năng : Lập
lịch trình tự động (Automated Itinerary Generation)
\hyperref[_Toc215924014]{23}}

\hyperref[_Toc215924015]{2.3.1 Dữ liệu đầu vào (Input)
\hyperref[_Toc215924015]{23}}

\hyperref[_Toc215924016]{2.3.2 Quy trình xử lý (Process)
\hyperref[_Toc215924016]{23}}

\hyperref[_Toc215924017]{2.3.3 Dữ liệu đầu ra (Output)
\hyperref[_Toc215924017]{24}}

\hyperref[_Toc215924018]{2.4 Đặc tả chức năng: Trợ lý ảo tư vấn (AI
Chatbot Consultation) \hyperref[_Toc215924018]{25}}

\hyperref[_Toc215924019]{2.4.1 Dữ liệu đầu vào (Input)
\hyperref[_Toc215924019]{25}}

\hyperref[_Toc215924020]{2.4.2 Quy trình xử lý (Process) Quy trình áp
dụng kỹ thuật RAG (Retrieval Augmented Generation):
\hyperref[_Toc215924020]{25}}

\hyperref[_Toc215924021]{2.4.3 Dữ liệu đầu ra (Output)
\hyperref[_Toc215924021]{25}}

\hyperref[_Toc215924022]{2.5 Đặc tả chức năng: Quản trị dữ liệu (Data
Management) \hyperref[_Toc215924022]{26}}

\hyperref[_Toc215924023]{2.5.1 Quản lý Địa điểm (CRUD Locations)
\hyperref[_Toc215924023]{26}}

\hyperref[_Toc215924024]{2.5.2 Quản lý Tri thức (Knowledge Base
Management) \hyperref[_Toc215924024]{26}}

\hyperref[_Toc215924025]{2.5.3 Thống kê và Giám sát (Monitoring)
\hyperref[_Toc215924025]{26}}

\hyperref[_Toc215924026]{2.6 Hiệu năng (Performance Efficiency)
\hyperref[_Toc215924026]{26}}

\hyperref[_Toc215924027]{2.6.1 Thời gian phản hồi (Response Time
Latency) \hyperref[_Toc215924027]{26}}

\hyperref[_Toc215924028]{2.6.2 Khả năng chịu tải (Throughput)
\hyperref[_Toc215924028]{27}}

\hyperref[_Toc215924029]{2.7 Tính tin cậy và Sẵn sàng (Reliability \&
Availability) \hyperref[_Toc215924029]{27}}

\hyperref[_Toc215924030]{2.7.1 Cơ chế hoạt động độc lập (Offline
Capability) \hyperref[_Toc215924030]{27}}

\hyperref[_Toc215924031]{2.7.2 Xử lý lỗi toàn cục (Global Error
Handling) \hyperref[_Toc215924031]{27}}

\hyperref[_Toc215924032]{2.8 An toàn và Bảo mật thông tin (Security)
\hyperref[_Toc215924032]{27}}

\hyperref[_Toc215924033]{2.8.1 Kiểm soát truy cập (Access Control)
\hyperref[_Toc215924033]{27}}

\hyperref[_Toc215924034]{2.8.2 Bảo mật dữ liệu nhạy cảm
\hyperref[_Toc215924034]{28}}

\hyperref[_Toc215924035]{2.9 Khả năng bảo trì và Mở rộng
(Maintainability \& Scalability) \hyperref[_Toc215924035]{28}}

\hyperref[_Toc215924036]{2.9.1 Kiến trúc Modular Monolith
\hyperref[_Toc215924036]{28}}

\hyperref[_Toc215924037]{2.9.2 Khả năng mở rộng Database
\hyperref[_Toc215924037]{28}}

\hyperref[_Toc215924038]{2.10 Quy trình nghiệp vụ Sinh lịch trình du
lịch (Itinerary Generation Process) \hyperref[_Toc215924038]{28}}

\hyperref[_Toc215924039]{2.11 Quy trình nghiệp vụ: Hội thoại thông minh
(Conversation Flow) \hyperref[_Toc215924039]{29}}

\hyperref[_Toc215924040]{2.12 Sơ đồ trạng thái của Địa điểm (Entity
State Diagram) \hyperref[_Toc215924040]{30}}

\hyperref[_Toc215924041]{2.13 Kiến trúc tổng thể (High-Level
Architecture) \hyperref[_Toc215924041]{32}}

\hyperref[_Toc215924042]{2.14 Thiết kế chi tiết kiến trúc Backend
(Layered Architecture) \hyperref[_Toc215924042]{33}}

\hyperref[_Toc215924043]{2.14.1 Lớp Định tuyến và Điều khiển (Controller
Layer) \hyperref[_Toc215924043]{33}}

\hyperref[_Toc215924044]{2.14.2 Lớp Dịch vụ (Service Layer)
\hyperref[_Toc215924044]{33}}

\hyperref[_Toc215924045]{2.14.3 Lớp Truy cập Dữ liệu (Data Access Layer
/ Repository) \hyperref[_Toc215924045]{33}}

\hyperref[_Toc215924046]{2.15 Thiết kế Kiến trúc Frontend
(Component-Based Architecture) \hyperref[_Toc215924046]{34}}

\hyperref[_Toc215924047]{2.16 Chiến lược Mô hình hóa Dữ liệu
\hyperref[_Toc215924047]{34}}

\hyperref[_Toc215924048]{2.17 Đặc tả chi tiết các Bảng dữ liệu (Table
Specifications) \hyperref[_Toc215924048]{34}}

\hyperref[_Toc215924049]{2.17.1 Bảng Location (Địa điểm) Đây là bảng
trung tâm (Core Table). \hyperref[_Toc215924049]{34}}

\hyperref[_Toc215924050]{2.17.2 Bảng Knowledge (Tri thức AI)
\hyperref[_Toc215924050]{35}}

\hyperref[_Toc215924051]{2.17.3 Bảng Admin (Quản trị viên)
\hyperref[_Toc215924051]{35}}

\hyperref[_Toc215924052]{2.17.4 Bảng AccessLog (Nhật ký truy cập)
\hyperref[_Toc215924052]{35}}

\hyperref[_Toc215924053]{2.18 Chiến lược Tối ưu hóa Hiệu năng Cơ sở dữ
liệu (Database Performance Strategy) \hyperref[_Toc215924053]{35}}

\hyperref[_Toc215924054]{2.18.1 Chiến lược đánh chỉ mục (Indexing
Strategy) \hyperref[_Toc215924054]{35}}

\hyperref[_Toc215924055]{2.18.2 Quản lý kết nối (Connection Management)
\hyperref[_Toc215924055]{36}}

\hyperref[_Toc215924056]{2.18.3 Tối ưu hóa truy vấn dữ liệu (Query
Optimization) \hyperref[_Toc215924056]{36}}

\hyperref[_Toc215924057]{2.19 Định nghĩa bài toán và Không gian tìm kiếm
\hyperref[_Toc215924057]{36}}

\hyperref[_Toc215924058]{2.19.1 Mô hình toán học:
\hyperref[_Toc215924058]{36}}

\hyperref[_Toc215924059]{2.19.2 Các ràng buộc cứng (Hard Constraints):
\hyperref[_Toc215924059]{36}}

\hyperref[_Toc215924060]{2.20 Thiết kế chi tiết Giải thuật 7 Pha (The
7-Phase Algorithm) \hyperref[_Toc215924060]{37}}

\hyperref[_Toc215924061]{2.21 Cơ chế Fallback và Xử lý ngoại lệ
(Fallback Mechanism) \hyperref[_Toc215924061]{38}}

\hyperref[_Toc215924062]{2.22 Kiến trúc Module AI (AI Module
Architecture) \hyperref[_Toc215924062]{38}}

\hyperref[_Toc215924063]{2.22.1 Tầng Adapter (Lớp thích ứng):
\hyperref[_Toc215924063]{39}}

\hyperref[_Toc215924064]{2.22.2 Cơ chế RAG (Retrieval Augmented
Generation): \hyperref[_Toc215924064]{39}}

\hyperref[_Toc215924065]{2.23 Kỹ thuật Prompt Engineering (Kỹ nghệ ra
lệnh cho AI) \hyperref[_Toc215924065]{39}}

\hyperref[_Toc215924066]{2.24 Quản lý lịch sử hội thoại (Conversation
History Management) \hyperref[_Toc215924066]{40}}

\hyperref[_Toc215924067]{2.25 Chiến lược thiết kế trải nghiệm (UX
Strategy) \hyperref[_Toc215924067]{40}}

\hyperref[_Toc215924068]{2.26 Thiết kế chi tiết các màn hình (Screen
Design) \hyperref[_Toc215924068]{40}}

\hyperref[_Toc215924069]{2.26.1 Màn hình Nhập liệu thông minh (Smart
Input Form) \hyperref[_Toc215924069]{40}}

\hyperref[_Toc215924070]{2.26.2 Màn hình Kết quả Lịch trình (Itinerary
Timeline) \hyperref[_Toc215924070]{41}}

\hyperref[_Toc215924071]{2.26.3 Giao diện Chatbot (Conversational UI)
\hyperref[_Toc215924071]{41}}

\hyperref[_Toc215924072]{2.27 Cơ chế Bảo mật đa lớp (Layered Security)
\hyperref[_Toc215924072]{41}}

\hyperref[_Toc215924073]{2.27.1 Bảo mật ở mức Ứng dụng (Application
Level) \hyperref[_Toc215924073]{41}}

\hyperref[_Toc215924074]{2.27.2 Bảo mật ở mức Quản trị (Admin Security)
\hyperref[_Toc215924074]{41}}

\hyperref[_Toc215924075]{2.28 Chiến lược Quản lý Lỗi (Error Handling
Strategy) \hyperref[_Toc215924075]{42}}

\hyperref[_Toc215924076]{2.28.1 Centralized Error Handling (Xử lý lỗi
tập trung) \hyperref[_Toc215924076]{42}}

\hyperref[_Toc215924077]{2.28.2 Fallback Logic (Logic dự phòng)
\hyperref[_Toc215924077]{42}}

\hyperref[citations]{CHƯƠNG 3: KẾT QUẢ VÀ ỨNG DỤNG
\hyperref[citations]{43}}

\hyperref[_Toc215924079]{3.1 Môi trường và công cụ phát triển
\hyperref[_Toc215924079]{43}}

\hyperref[_Toc215924080]{3.1.1 Cấu hình phần cứng (Development
Environment) \hyperref[_Toc215924080]{43}}

\hyperref[_Toc215924081]{3.1.2 Nền tảng và Thư viện Phần mềm Danh sách
các công nghệ cốt lõi được sử dụng trong mã nguồn:
\hyperref[_Toc215924081]{43}}

\hyperref[_Toc215924082]{3.2 Quy trình cài đặt và triển khai (deployment
guide) \hyperref[_Toc215924082]{43}}

\hyperref[_Toc215924083]{3.3 Kết quả hiện thực hóa các chức năng
(detailed implementation) \hyperref[_Toc215924083]{44}}

\hyperref[_Toc215924084]{3.3.1 Phân hệ Khách du lịch - Trang chủ (Home
Page) \hyperref[_Toc215924084]{44}}

\hyperref[_Toc215924085]{3.3.2 Phân hệ Khách du lịch - Lập lịch trình
thông minh (Smart Itinerary) \hyperref[_Toc215924085]{45}}

\hyperref[_Toc215924086]{3.3.3 Phân hệ Khách du lịch - Trợ lý ảo AI (RAG
Chatbot) \hyperref[_Toc215924086]{47}}

\hyperref[_Toc215924087]{3.3.4 Phân hệ Quản trị viên (Admin Dashboard)
\hyperref[_Toc215924087]{48}}

\hyperref[_Toc215924088]{3.4 KỊCH BẢN KIỂM THỬ VÀ KẾT QUẢ (SYSTEM
TESTING) \hyperref[_Toc215924088]{49}}

\hyperref[_Toc215924089]{3.5 Đánh giá hiệu năng (performance evaluation)
\hyperref[_Toc215924089]{50}}

\hyperref[_Toc215924090]{3.5.1 Hiệu năng phía Máy chủ (Backend Latency)
\hyperref[_Toc215924090]{51}}

\hyperref[_Toc215924091]{3.5.2 Hiệu năng phía Máy khách (Frontend
Metrics) \hyperref[_Toc215924091]{51}}

\hyperref[kux1ebft-luux1eadn]{KẾT LUẬN
\hyperref[kux1ebft-luux1eadn]{52}}

\hyperref[tuxe0i-liux1ec7u-tham-khux1ea3o]{TÀI LIỆU THAM KHẢO
\hyperref[tuxe0i-liux1ec7u-tham-khux1ea3o]{53}}

\hyperref[phux1ee5-lux1ee5c]{PHỤ LỤC \hyperref[phux1ee5-lux1ee5c]{54}}

\section{\texorpdfstring{\hfill\break
MỞ ĐẦU}{ MỞ ĐẦU}}\label{mux1edf-ux111ux1ea7u}

\begin{enumerate}
\def\labelenumi{\arabic{enumi}.}
\item
  \textbf{Lý do chọn đề tài}
\end{enumerate}

Trong bối cảnh du lịch Đà Nẵng đang phục hồi và phát triển mạnh mẽ sau
đại dịch, nhu cầu tìm kiếm thông tin và lên kế hoạch cho chuyến đi của
du khách ngày càng tăng cao. Tuy nhiên, em nhận thấy một vấn đề thực tế
(pain point) mà rất nhiều du khách, đặc biệt là các nhóm bạn trẻ hoặc
khách du lịch tự túc gặp phải: đó là sự quá tải thông tin ("information
overload"). Có quá nhiều bài review trên mạng xã hội, quá nhiều địa điểm
ăn uống, vui chơi khiến du khách bối rối không biết sắp xếp lịch trình
sao cho hợp lý về mặt thời gian và địa lý.

Hiện nay đã có các công cụ hỗ trợ như Google Maps hay TripAdvisor, nhưng
chúng chỉ dừng lại ở việc tra cứu thông tin rời rạc. Chưa có nhiều ứng
dụng miễn phí, đơn giản cho phép du khách nhập vào số ngày đi và sở
thích, sau đó hệ thống tự động trả về một lịch trình chi tiết từng khung
giờ (Sáng - Trưa - Chiều - Tối) mà không cần đăng nhập phức tạp.

Xuất phát từ nhu cầu thực tế đó, cùng với mong muốn áp dụng các kiến
thức đã học về công nghệ Web và Trí tuệ nhân tạo (Generative AI) vào
thực tiễn, em quyết định lựa chọn đề tài: \textbf{"Xây dựng hệ thống hỗ
trợ lập lịch trình du lịch Đà Nẵng tích hợp Chatbot tư vấn thông minh
(Dana Travel)"}.

\begin{enumerate}
\def\labelenumi{\arabic{enumi}.}
\setcounter{enumi}{1}
\item
  \textbf{Mục tiêu của đề tài}
\end{enumerate}

Đề tài tập trung giải quyết hai bài toán cốt lõi:

\begin{itemize}
\item
  Thứ nhất: Xây dựng một công cụ lập lịch trình tự động (Itinerary
  Generator). Hệ thống phải có khả năng sắp xếp các địa điểm tham quan,
  ăn uống, vui chơi vào các khung giờ phù hợp trong ngày dựa trên một
  tập luật cố định (Rule-based) để đảm bảo tính khả thi cho chuyến đi
  (Ví dụ: Không thể đi Bà Nà Hills vào buổi tối, hoặc phải ăn trưa sau
  khi tham quan xong).
\item
  Thứ hai: Tích hợp một trợ lý ảo (AI Chatbot) sử dụng mô hình ngôn ngữ
  lớn (LLM) để tương tác tự nhiên với người dùng, giải đáp các thắc mắc
  nằm ngoài kịch bản có sẵn và đưa ra các gợi ý mang tính cá nhân hóa.
\end{itemize}

\begin{enumerate}
\def\labelenumi{\arabic{enumi}.}
\setcounter{enumi}{2}
\item
  \textbf{Đối tượng và phạm vi nghiên cứu}
\end{enumerate}

\begin{itemize}
\item
  Đối tượng nghiên cứu:

  \begin{itemize}
  \item
    Các công nghệ phát triển Web hiện đại: ReactJS (Frontend), Node.js
    (Backend).
  \item
    Hệ quản trị cơ sở dữ liệu gọn nhẹ SQLite và công cụ ORM Prisma.
  \item
    Kỹ thuật Prompt Engineering để tích hợp Google Gemini API.
  \item
    Các thuật toán sắp xếp lịch trình dựa trên luật (Rule-based
    Scheduling).
  \end{itemize}
\item
  Phạm vi nghiên cứu:

  \begin{itemize}
  \item
    Hệ thống tập trung vào dữ liệu du lịch tại thành phố Đà Nẵng và một
    số khu vực lân cận (Hội An).
  \item
    Về mặt người dùng: Hệ thống hướng tới sự tiện lợi tối đa, cho phép
    khách du lịch sử dụng ngay các tính năng chính mà không cần đăng ký
    tài khoản (Anonymous User). Chức năng quản trị (Admin) sẽ yêu cầu
    xác thực bảo mật.
  \end{itemize}
\end{itemize}

\begin{enumerate}
\def\labelenumi{\arabic{enumi}.}
\setcounter{enumi}{3}
\item
  \textbf{Phương pháp nghiên cứu}
\end{enumerate}

\begin{itemize}
\item
  Phương pháp tham khảo tài liệu: Nghiên cứu tài liệu chính hãng
  (Documentation) của React, Node.js, Prisma và Google AI Studio.
\item
  Phương pháp phân tích thiết kế hệ thống: Sử dụng mô hình phân lớp
  (Layered Architecture) để tổ chức mã nguồn, phân tích yêu cầu chức
  năng và phi chức năng.
\item
  Phương pháp thực nghiệm: Xây dựng ứng dụng thực tế, tiến hành kiểm thử
  các kịch bản sinh lịch trình và hội thoại với Chatbot để đánh giá độ
  chính xác.
\end{itemize}

\begin{enumerate}
\def\labelenumi{\arabic{enumi}.}
\setcounter{enumi}{4}
\item
  \textbf{Bố cục báo cáo}
\end{enumerate}

Ngoài phần mở đầu và kết luận, báo cáo được chia thành 3 chương:

\begin{itemize}
\item
  Chương 1: Cơ sở lý thuyết. Trình bày các nền tảng công nghệ và lý
  thuyết thuật toán được sử dụng.
\item
  Chương 2: Phát biểu vấn đề. Phân tích chi tiết bài toán, đặc tả yêu
  cầu và thiết kế hệ thống (CSDL, Kiến trúc).
\item
  Chương 3: Kết quả và Ứng dụng. Trình bày kết quả cài đặt chương trình,
  demo giao diện và các kịch bản kiểm thử.
\end{itemize}

\textbf{\hfill\break
}

\newpage
\section{CHƯƠNG 1: CƠ SỞ LÝ
THUYẾT}\label{chux1b0ux1a1ng-1-cux1a1-sux1edf-luxfd-thuyux1ebft}

\begin{enumerate}
\def\labelenumi{\arabic{enumi}.}
\item
  \protect\phantomsection\label{_Toc215923986}{}\textbf{Các chuẩn web
  nền tảng (Web Standards)}

  \begin{enumerate}
  \def\labelenumii{\arabic{enumii}.}
  \item
    \protect\phantomsection\label{_Toc215923987}{}\textbf{Tổng quan về
    HTML 5}
  \end{enumerate}
\end{enumerate}

HTML (HyperText Markup Language) là ngôn ngữ đánh dấu tiêu chuẩn để tạo
ra cấu trúc của các trang web. Phiên bản HTML5 là phiên bản mới nhất,
mang lại nhiều cải tiến vượt bậc về khả năng xử lý đa phương tiện và ngữ
nghĩa học (Semantics).

\includegraphics[width=1.71667in,height=1.71667in,alt={W3C HTML5 Logo}]{media/image5.png}

HTML5 không chỉ đơn thuần là định dạng văn bản mà đã phát triển thành
một nền tảng ứng dụng mạnh mẽ. Cấu trúc một tài liệu HTML bao gồm các
phần tử ngữ nghĩa (semantic elements) như \textless header\textgreater,
\textless nav\textgreater, \textless main\textgreater,
\textless footer\textgreater. Các thẻ này giúp trình duyệt và các công
cụ tìm kiếm (SEO) hiểu rõ ý nghĩa của nội dung, đồng thời cải thiện khả
năng tiếp cận (Accessibility) cho người dùng khuyết tật.

Trong các ứng dụng hiện đại sử dụng ReactJS, HTML không được viết trực
tiếp mà thông qua cú pháp JSX (JavaScript XML). JSX cho phép viết cấu
trúc HTML ngay trong mã JavaScript, giúp việc xây dựng các thành phần
giao diện (Components) trở nên trực quan và gắn kết chặt chẽ với logic
xử lý.

\begin{enumerate}
\def\labelenumi{\arabic{enumi}.}
\setcounter{enumi}{1}
\item
  \protect\phantomsection\label{_Toc215923988}{}\textbf{Tổng quan về
  CSS}
\end{enumerate}

CSS (Cascading Style Sheets) là ngôn ngữ dùng để mô tả cách hiển thị của
các phần tử HTML trên màn hình, giấy in hoặc các phương tiện khác. CSS3
là phiên bản nâng cấp với các tính năng đồ họa mạnh mẽ như đổ bóng, bo
tròn, hiệu ứng chuyển động (animation) mà không cần dùng đến hình ảnh
hay Flash.

\includegraphics[width=1.38333in,height=1.38333in,alt={Css - Free brands and logotypes icons}]{media/image6.png}

Nếu HTML là khung xương thì CSS chính là "lớp da" và "lớp trang điểm",
quyết định giao diện thẩm mỹ của ứng dụng. Một yêu cầu bắt buộc của đồ
án là ứng dụng phải chạy tốt trên cả điện thoại và máy tính. Em sử dụng
CSS Media Queries để tự động điều chỉnh kích thước, bố cục giao diện dựa
trên độ rộng màn hình của thiết bị người dùng. Trong dự án, các file cấu
hình như postcss.config.js giúp xử lý các quy tắc CSS hiện đại để tương
thích với nhiều trình duyệt khác nhau.

\begin{enumerate}
\def\labelenumi{\arabic{enumi}.}
\setcounter{enumi}{1}
\item
  \protect\phantomsection\label{_Toc215923989}{}\textbf{Công nghệ phát
  triển Giao diện (Frontend)}
\end{enumerate}

Để vượt qua giới hạn của các trang web tĩnh truyền thống, em sử dụng các
thư viện và framework hiện đại để xây dựng một ứng dụng web động
(Dynamic Web App).

\begin{enumerate}
\def\labelenumi{\arabic{enumi}.}
\item
  \protect\phantomsection\label{_Toc215923990}{}\textbf{Ngôn ngữ lập
  trình JavaScript (ES6+)}
\end{enumerate}

JavaScript được Brendan Eich tạo ra tại Netscape năm 1995. Từ một ngôn
ngữ kịch bản đơn giản chạy trên trình duyệt, JavaScript đã lột xác nhờ
sự chuẩn hóa ECMAScript (ES). Phiên bản ES6 (2015) mang đến những cải
tiến đột phá như: từ khóa let/const để quản lý biến an toàn, Arrow
Function giúp viết hàm ngắn gọn, và Modules để tổ chức mã nguồn khoa học
hơn.

\includegraphics[width=3.44444in,height=1.37778in,alt={javaScript ES6 2015 what you need to know as a Developer?}]{media/image7.png}

Đặc điểm quan trọng nhất của JavaScript là mô hình Đơn luồng
(Single-thread) kết hợp với cơ chế Vòng lặp sự kiện (Event Loop)

Cơ chế này cho phép JavaScript xử lý các tác vụ bất đồng bộ (như gọi
API, đọc file) mà không làm "đóng băng" giao diện người dùng, thông qua
các kỹ thuật như Callback, Promise và cú pháp hiện đại Async/Await.

Trong dự án này, JavaScript đóng vai trò là ngôn ngữ chủ đạo xuyên suốt
từ phía máy khách (Frontend) đến phía máy chủ (Backend), tạo nên sự đồng
nhất về mặt ngữ pháp và tư duy lập trình (Full-stack JavaScript).

\begin{enumerate}
\def\labelenumi{\arabic{enumi}.}
\setcounter{enumi}{1}
\item
  \protect\phantomsection\label{_Toc215923991}{}\textbf{Tổng quan về
  ReactJS}
\end{enumerate}

ReactJS là một thư viện JavaScript mã nguồn mở được Facebook (Meta) phát
triển, dùng để xây dựng giao diện người dùng (UI). Khác với cách thao
tác DOM truyền thống (như jQuery), React tiếp cận theo hướng Dựa trên
thành phần (Component-based). Giao diện được chia nhỏ thành các thành
phần độc lập (như Header, Footer, Card), có khả năng tái sử dụng cao.

\includegraphics[width=3.65in,height=1.825in,alt={Reactjs logo - Free Icon PNG, SVG}]{media/image8.png}

Cơ sở lý thuyết quan trọng của React là DOM ảo (Virtual DOM). Thay vì
cập nhật trực tiếp lên DOM thật (rất tốn tài nguyên), React tạo ra một
bản sao của DOM trong bộ nhớ. Khi dữ liệu thay đổi, React so sánh sự
khác biệt (Diffing) và chỉ cập nhật những phần thực sự thay đổi lên màn
hình. Điều này giúp tối ưu hóa hiệu năng render.

React cũng phổ biến khái niệm SPA (Single Page Application). Ứng dụng
chỉ tải một trang HTML duy nhất, sau đó nội dung được cập nhật động mà
không cần tải lại trang, mang lại trải nghiệm mượt mà như ứng dụng
Desktop.

\begin{enumerate}
\def\labelenumi{\arabic{enumi}.}
\setcounter{enumi}{2}
\item
  \protect\phantomsection\label{_Toc215923992}{}\textbf{Framework
  Tailwind CSS}
\end{enumerate}

Tailwind CSS là một framework CSS mã nguồn mở, được phát triển với triết
lý utility-first, giúp xây dựng giao diện người dùng (UI) nhanh chóng và
linh hoạt. Khác với cách viết CSS truyền thống (như Bootstrap với các
component dựng sẵn), Tailwind tiếp cận theo hướng dựa trên lớp tiện ích
(Utility-based). Giao diện được tạo ra bằng cách kết hợp các class nhỏ
gọn (như flex, text-center, bg-blue-500), cho phép tùy biến cao và tái
sử dụng dễ dàng.

\includegraphics[width=1.44167in,height=1.44167in,alt={Tailwind CSS Logo Rounded 67565433 PNG}]{media/image9.png}

Cơ sở lý thuyết quan trọng của Tailwind là utility classes\textbf{.}
Thay vì định nghĩa CSS riêng cho từng thành phần, Tailwind cung cấp sẵn
hàng trăm lớp tiện ích mô tả trực tiếp các thuộc tính CSS (màu sắc,
khoảng cách, bố cục\ldots). Khi cần thay đổi giao diện, lập trình viên
chỉ việc chỉnh sửa class trong HTML, giúp giảm thiểu việc viết CSS thủ
công và tối ưu hóa tốc độ phát triển.

Tailwind cũng phổ biến khái niệm Responsive Design. Framework này tích
hợp sẵn hệ thống breakpoint (sm:, md:, lg:, xl:), cho phép giao diện tự
động thích ứng trên nhiều kích thước màn hình mà không cần viết media
query phức tạp. Ngoài ra, Tailwind hỗ trợ tùy biến theme (màu sắc, font,
spacing\ldots) thông qua file cấu hình, giúp dễ dàng đồng bộ hóa thiết
kế trong toàn bộ dự án.

\begin{enumerate}
\def\labelenumi{\arabic{enumi}.}
\setcounter{enumi}{2}
\item
  \protect\phantomsection\label{_Toc215923993}{}\textbf{Công nghệ phát
  triển Hệ thống máy chủ (Backend)}

  \begin{enumerate}
  \def\labelenumii{\arabic{enumii}.}
  \item
    \protect\phantomsection\label{_Toc215923994}{}\textbf{Tổng quan về
    Node.js}
  \end{enumerate}
\end{enumerate}

Node.js là một môi trường chạy JavaScript mã nguồn mở, được xây dựng
trên V8 Engine của Google Chrome. Khác với JavaScript truyền thống vốn
chỉ chạy trong trình duyệt, Node.js cho phép lập trình viên chạy
JavaScript ở phía máy chủ (server-side). Điều này mở rộng khả năng của
JavaScript, biến nó thành một ngôn ngữ lập trình toàn diện cho cả
frontend và backend.

\includegraphics[width=1.8366in,height=1.125in,alt={A logo of a company AI-generated content may be incorrect.}]{media/image10.png}

Cơ sở lý thuyết quan trọng của Node.js là mô hình bất đồng bộ
(asynchronous) và hướng sự kiện (event-driven). Thay vì xử lý tuần tự và
chặn luồng (blocking), Node.js sử dụng cơ chế non-blocking I/O. Khi có
tác vụ như đọc file, truy vấn cơ sở dữ liệu hay gọi API, Node.js không
chờ kết quả mà tiếp tục xử lý các tác vụ khác, sau đó nhận phản hồi
thông qua callback hoặc Promise. Cách tiếp cận này giúp Node.js có hiệu
năng cao, đặc biệt phù hợp cho các ứng dụng cần xử lý nhiều kết nối đồng
thời.

Node.js cũng phổ biến khái niệm Single-threaded Event Loop. Mặc dù chỉ
chạy trên một luồng duy nhất, Node.js có thể quản lý hàng nghìn kết nối
nhờ cơ chế event loop và thư viện libuv. Đây là điểm khác biệt lớn so
với các server truyền thống vốn tạo nhiều luồng (multi-thread) cho mỗi
kết nối, dễ gây tốn tài nguyên.

Ngoài ra, Node.js đi kèm với npm (Node Package Manager) -- hệ sinh thái
thư viện khổng lồ, giúp lập trình viên dễ dàng cài đặt, chia sẻ và tái
sử dụng các module. Nhờ đó, Node.js trở thành nền tảng lý tưởng để xây
dựng ứng dụng web thời gian thực (real-time) như chat app, game online,
API server, và cả các hệ thống microservices.

\begin{enumerate}
\def\labelenumi{\arabic{enumi}.}
\setcounter{enumi}{1}
\item
  \protect\phantomsection\label{_Toc215923995}{}\textbf{Framework
  Express.js}
\end{enumerate}

Express.js là một framework web mã nguồn mở dành cho Node.js, được thiết
kế để xây dựng ứng dụng web và API một cách nhanh chóng, linh hoạt và
tối giản. Khác với việc tự viết toàn bộ logic xử lý HTTP trong Node.js,
Express cung cấp một lớp trừu tượng đơn giản, giúp lập trình viên dễ
dàng định nghĩa tuyến đường (routes), middleware, và xử lý yêu cầu/ phản
hồi.

\includegraphics[width=2.73333in,height=1.19651in,alt={About Express.js. Express.js is a Web server framework\ldots{} \textbar{} by Rahul Fernando \textbar{} Medium}]{media/image11.png}

Cơ sở lý thuyết quan trọng của Express là middleware. Middleware là các
hàm trung gian nằm trong chuỗi xử lý request--response. Mỗi middleware
có thể thực hiện một tác vụ cụ thể (như xác thực, logging, parse dữ liệu
JSON, xử lý lỗi), sau đó chuyển tiếp yêu cầu sang middleware tiếp theo.
Cách tiếp cận này giúp ứng dụng dễ mở rộng, dễ bảo trì và có cấu trúc rõ
ràng.

Express cũng phổ biến khái niệm Routing. Thay vì phải viết logic phức
tạp để phân tích URL, Express cho phép định nghĩa các tuyến đường theo
cú pháp đơn giản, ví dụ:

app.get(\textquotesingle/users\textquotesingle, (req, res)
=\textgreater{} \{

~ res.send(\textquotesingle Danh sách người dùng\textquotesingle);

\});

Điều này giúp quản lý API và các endpoint trong ứng dụng một cách trực
quan.

Ngoài ra, Express hỗ trợ tích hợp dễ dàng với nhiều template engine (như
Pug, EJS, Handlebars) để render HTML động, và có khả năng kết nối với cơ
sở dữ liệu (MongoDB, MySQL, PostgreSQL\ldots) thông qua các thư viện bên
ngoài. Nhờ hệ sinh thái npm phong phú, Express có thể mở rộng để phục vụ
từ ứng dụng nhỏ đến hệ thống lớn.

\begin{enumerate}
\def\labelenumi{\arabic{enumi}.}
\setcounter{enumi}{2}
\item
  \protect\phantomsection\label{_Toc215923996}{}\textbf{Chuẩn RESTful
  API}
\end{enumerate}

RESTful API (Representational State Transfer) là một chuẩn kiến trúc phổ
biến trong việc xây dựng hệ thống giao tiếp giữa Client và Server. Khác
với cách truyền dữ liệu phức tạp hoặc giữ trạng thái trên server, REST
tiếp cận theo hướng đơn giản, phi trạng thái (Stateless) và tận dụng
đúng ngữ nghĩa của giao thức HTTP.

\includegraphics[width=1.87535in,height=1.44444in,alt={REST API logo on white background}]{media/image12.jpeg}

Cơ sở lý thuyết quan trọng của REST là tính phi trạng thái (Stateless).
Mỗi request từ client phải chứa đầy đủ thông tin cần thiết (ví dụ: dữ
liệu, token xác thực). Server không lưu trạng thái của client giữa các
request, nhờ đó hệ thống dễ mở rộng, giảm tải bộ nhớ và tăng khả năng xử
lý song song.

REST cũng phổ biến khái niệm phương thức chuẩn HTTP. Mỗi hành động được
ánh xạ trực tiếp với một phương thức HTTP:

\begin{itemize}
\item
  GET → Lấy dữ liệu (ví dụ: danh sách người dùng).
\item
  POST → Tạo mới dữ liệu (ví dụ: thêm lịch trình).
\item
  PUT → Cập nhật dữ liệu hiện có.
\item
  DELETE → Xóa dữ liệu.
\end{itemize}

Ngoài ra, REST định nghĩa tài nguyên (resource) thông qua URL rõ ràng.
Ví dụ:

GET /api/users ~ ~ ~ ~→ Lấy danh sách người dùng

POST /api/schedules ~ → Tạo mới lịch trình

PUT /api/users/123 ~ ~→ Cập nhật thông tin user có ID = 123

DELETE /api/users/123 → Xóa user có ID = 123

REST thường sử dụng \textbf{JSON} làm định dạng dữ liệu truyền tải, nhờ
tính gọn nhẹ, dễ đọc và dễ tích hợp với nhiều nền tảng khác nhau (web,
mobile, IoT\ldots).

\begin{enumerate}
\def\labelenumi{\arabic{enumi}.}
\item
  \protect\phantomsection\label{_Toc215923997}{}\textbf{Hệ quản trị Cơ
  sở dữ liệu (Database)}

  \begin{enumerate}
  \def\labelenumii{\arabic{enumii}.}
  \item
    \protect\phantomsection\label{_Toc215923998}{}\textbf{SQLite}
  \end{enumerate}
\end{enumerate}

SQLite là một hệ quản trị cơ sở dữ liệu quan hệ (RDBMS) mã nguồn mở, gọn
nhẹ và tự chứa (self-contained). Khác với các hệ quản trị cơ sở dữ liệu
truyền thống như MySQL hay PostgreSQL vốn hoạt động theo mô hình
client--server, SQLite tiếp cận theo hướng embedded database -- toàn bộ
cơ sở dữ liệu được lưu trữ trong một file duy nhất trên hệ thống

\includegraphics[width=1.16667in,height=1.16667in,alt={sqlite Vector Icons free download in SVG, PNG Format}]{media/image13.png}

Cơ sở lý thuyết quan trọng của SQLite là tính đơn giản và gọn nhẹ. Không
cần cài đặt phức tạp hay chạy dịch vụ nền, SQLite có thể được tích hợp
trực tiếp vào ứng dụng. Điều này giúp lập trình viên dễ dàng triển khai
cơ sở dữ liệu mà không cần quản lý server riêng biệt.

SQLite cũng phổ biến khái niệm file-based database. Toàn bộ dữ liệu
(bảng, chỉ mục, trigger\ldots) được lưu trong một file .sqlite hoặc .db.
Việc sao lưu, di chuyển hay chia sẻ cơ sở dữ liệu chỉ đơn giản là copy
file này.

Ngoài ra, SQLite vẫn tuân thủ chuẩn SQL-92, hỗ trợ đầy đủ các câu lệnh
SQL như SELECT, INSERT, UPDATE, DELETE, cùng với các tính năng như
transaction, foreign key, và indexing. Mặc dù không mạnh mẽ bằng các hệ
quản trị lớn trong xử lý dữ liệu khổng lồ, SQLite lại cực kỳ hiệu quả
cho ứng dụng vừa và nhỏ, hoặc các hệ thống nhúng (embedded systems).

\begin{itemize}
\item
  Ưu điểm

  \begin{itemize}
  \item
    Gọn nhẹ: Không cần cài đặt phức tạp, chỉ cần một file duy nhất.
  \item
    Nhanh chóng: Tối ưu cho ứng dụng có lượng dữ liệu vừa phải.
  \item
    Dễ tích hợp: Phù hợp cho ứng dụng desktop, mobile, IoT.
  \item
    Đa nền tảng: Hoạt động trên hầu hết hệ điều hành.
  \end{itemize}
\item
  Hạn chế

  \begin{itemize}
  \item
    Không phù hợp cho hệ thống phân tán hoặc dữ liệu cực lớn.
  \item
    Thiếu các tính năng nâng cao như replication, clustering.

    \begin{enumerate}
    \def\labelenumi{\arabic{enumi}.}
    \item
      \protect\phantomsection\label{_Toc215923999}{}\textbf{Prisma ORM
      (Object-Relational Mapping)}
    \end{enumerate}
  \end{itemize}
\end{itemize}

Prisma là một ORM (Object-Relational Mapping) hiện đại dành cho Node.js
và TypeScript, giúp lập trình viên tương tác với cơ sở dữ liệu một cách
trực quan, an toàn và hiệu quả. Khác với việc viết truy vấn SQL thủ
công, Prisma tiếp cận theo hướng type-safe và declarative, cho phép định
nghĩa mô hình dữ liệu trong file schema, sau đó tự động sinh ra client
để thao tác với database.

\includegraphics[width=1.19167in,height=1.19167in,alt={Prisma Orm color icon in PNG, SVG}]{media/image14.jpeg}

Cơ sở lý thuyết quan trọng của Prisma là Prisma Schema. Đây là file
trung tâm (schema.prisma) nơi lập trình viên định nghĩa các bảng
(model), quan hệ (relation), và cấu hình kết nối cơ sở dữ liệu. Từ
schema này, Prisma sẽ sinh ra Prisma Client -- một thư viện tự động, cho
phép gọi hàm như prisma.user.findMany() thay vì viết câu lệnh SQL phức
tạp.

Prisma cũng phổ biến khái niệm type-safety. Vì Prisma được tích hợp chặt
chẽ với TypeScript, mọi truy vấn đều được kiểm tra kiểu dữ liệu ngay
trong quá trình viết code. Điều này giúp giảm thiểu lỗi runtime, tăng độ
tin cậy và năng suất phát triển.

Ngoài ra, Prisma hỗ trợ Migration System để quản lý thay đổi cấu trúc cơ
sở dữ liệu. Khi cập nhật schema, Prisma có thể tự động tạo migration
script, đảm bảo database luôn đồng bộ với mô hình dữ liệu trong code.

\begin{itemize}
\item
  Ưu điểm

  \begin{itemize}
  \item
    Type-safe: Tích hợp với TypeScript, giảm thiểu lỗi khi truy vấn dữ
    liệu.
  \item
    Dễ sử dụng: API trực quan, dễ đọc, thay thế cho SQL phức tạp.
  \item
    Hỗ trợ nhiều DB: Tương thích với PostgreSQL, MySQL, SQLite,
    MongoDB\ldots{}
  \item
    Migration tiện lợi: Quản lý thay đổi schema dễ dàng.
  \item
    Hiện đại: Thiết kế phù hợp với hệ sinh thái JavaScript/TypeScript.
  \end{itemize}
\item
  Hạn chế

  \begin{itemize}
  \item
    Phụ thuộc vào hệ sinh thái Node.js/TypeScript.
  \item
    Không phù hợp cho những dự án cần tối ưu truy vấn SQL phức tạp ở mức
    thấp.
  \end{itemize}

  \begin{enumerate}
  \def\labelenumi{\arabic{enumi}.}
  \item
    \protect\phantomsection\label{_Toc215924000}{}\textbf{Trí tuệ nhân
    tạo và Mô hình ngôn ngữ lớn (LLM)}

    \begin{enumerate}
    \def\labelenumii{\arabic{enumii}.}
    \item
      \protect\phantomsection\label{_Toc215924001}{}\textbf{Mô hình
      Generative AI (Google Gemini)}
    \end{enumerate}
  \end{enumerate}
\end{itemize}

Google Gemini là mô hình Generative AI thế hệ mới của Google DeepMind,
được thiết kế để xử lý đa phương thức (multimodal) -- nghĩa là có khả
năng hiểu và tạo nội dung từ văn bản, hình ảnh, âm thanh, và thậm chí cả
mã nguồn. Đây là bước tiến lớn trong việc xây dựng trí tuệ nhân tạo toàn
diện, vượt xa các mô hình chỉ dựa trên văn bản

\includegraphics[width=1.8in,height=0.4067in,alt={A black background with a black square AI-generated content may be incorrect.}]{media/image15.png}

Gemini là dòng mô hình AI tạo sinh (Generative AI) của Google, kế thừa
từ Bard và phát triển bởi Google DeepMind.

Nó được thiết kế để trở thành AI đa năng, có thể hỗ trợ viết, lập kế
hoạch, phân tích dữ liệu, sáng tạo nội dung, và tương tác tự nhiên với
người dùng.

\begin{enumerate}
\def\labelenumi{\arabic{enumi}.}
\setcounter{enumi}{1}
\item
  \protect\phantomsection\label{_Toc215924002}{}\textbf{Kỹ thuật RAG
  (Retrieval-Augêmnted Generation)}
\end{enumerate}

RAG (Retrieval-Augmented Generation) là một kỹ thuật trong lĩnh vực
Generative AI, kết hợp giữa khả năng tìm kiếm thông tin (retrieval) và
khả năng sinh ngôn ngữ tự nhiên (generation). Khác với mô hình ngôn ngữ
truyền thống vốn chỉ dựa vào dữ liệu đã được huấn luyện, RAG tiếp cận
theo hướng bổ sung kiến thức từ nguồn dữ liệu bên ngoài, giúp câu trả
lời chính xác, cập nhật và giàu ngữ cảnh hơn.

\includegraphics[width=1.86667in,height=1.86667in,alt={What is Retrieval-Augmented Generation (RAG)? \textbar{} Google Cloud}]{media/image16.png}

Cơ sở lý thuyết quan trọng của RAG là sự kết hợp hai thành phần:

\begin{itemize}
\item
  Retrieval (Truy xuất): Khi nhận câu hỏi từ người dùng, hệ thống sẽ tìm
  kiếm thông tin liên quan từ cơ sở dữ liệu, tài liệu, hoặc kho tri thức
  bên ngoài.
\item
  Generation (Sinh ngôn ngữ): Sau khi có thông tin, mô hình ngôn ngữ sẽ
  tổng hợp và diễn đạt lại thành câu trả lời tự nhiên, mạch lạc, dễ
  hiểu.
\end{itemize}

RAG cũng phổ biến khái niệm augmenting knowledge tức là tăng cường kiến
thức cho mô hình. Thay vì chỉ dựa vào dữ liệu huấn luyện tĩnh (có thể
lỗi thời), RAG cho phép mô hình truy cập thông tin mới nhất, đảm bảo câu
trả lời luôn cập nhật và phù hợp với ngữ cảnh thực tế.

Ví dụ: Khi người dùng hỏi về ``thông tin vé, giờ mở cửa'', một mô hình
ngôn ngữ thuần túy có thể không biết. Nhưng với RAG, hệ thống sẽ truy
xuất dữ liệu từ nguồn tin tức, sau đó sinh câu trả lời chính xác và tự
nhiên.

\begin{enumerate}
\def\labelenumi{\arabic{enumi}.}
\item
  \protect\phantomsection\label{_Toc215924003}{}\textbf{Lý thuyết bài
  toán xếp lịch (Scheduling)}

  \begin{enumerate}
  \def\labelenumii{\arabic{enumii}.}
  \item
    \protect\phantomsection\label{_Toc215924004}{}\textbf{Bài toán thỏa
    mãn ràng buộc (CSP)}
  \end{enumerate}
\end{enumerate}

Bài toán thỏa mãn ràng buộc (CSP -- Constraint Satisfaction Problem) là
một mô hình toán học trong trí tuệ nhân tạo và khoa học máy tính, dùng
để biểu diễn và giải quyết các vấn đề có tập biến (variables) và tập
ràng buộc (constraints). Khác với cách giải quyết tuần tự hoặc
brute-force, CSP tiếp cận theo hướng khai thác cấu trúc ràng buộc để thu
hẹp không gian tìm kiếm, từ đó tìm ra nghiệm hợp lệ một cách hiệu quả.

Cơ sở lý thuyết quan trọng của CSP là biểu diễn vấn đề bằng ba thành
phần chính:

\begin{itemize}
\item
  Tập biến (Variables): Các đại lượng cần tìm giá trị (ví dụ: màu của
  các vùng trên bản đồ, thời gian cho các lớp học).
\item
  Miền giá trị (Domains): Tập hợp các giá trị có thể gán cho mỗi biến
  (ví dụ: \{Đỏ, Xanh, Vàng\} cho bài toán tô màu bản đồ).
\item
  Tập ràng buộc (Constraints): Các điều kiện mà nghiệm phải thỏa mãn (ví
  dụ: hai vùng kề nhau không được cùng màu).
\end{itemize}

CSP cũng phổ biến khái niệm tìm kiếm có ràng buộc (Constraint-based
search). Thay vì thử tất cả khả năng, thuật toán sẽ loại bỏ các giá trị
không hợp lệ ngay từ đầu (forward checking, arc consistency), giúp giảm
đáng kể số lượng trạng thái cần xét.

Ví dụ điển hình của CSP:

\begin{itemize}
\item
  Ràng buộc thời gian (Time Windows): Địa điểm phải mở cửa trong giờ
  tham quan.
\item
  Ràng buộc loại hình (Category Constraints): Buổi trưa/tối bắt buộc là
  Ăn uống (FOOD), Sáng/Chiều là Tham quan (ATTRACTION/BEACH).

  \begin{enumerate}
  \def\labelenumi{\arabic{enumi}.}
  \item
    \protect\phantomsection\label{_Toc215924005}{}\textbf{Giải thuật
    Heuristic (Heuristic Algorithm)}
  \end{enumerate}
\end{itemize}

Giải thuật Heuristic (Heuristic Algorithm) là một phương pháp giải quyết
vấn đề dựa trên kinh nghiệm, quy tắc phỏng đoán hoặc chiến lược gần
đúng, thay vì tìm kiếm lời giải tối ưu tuyệt đối. Khác với các giải
thuật chính xác (exact algorithms) vốn đảm bảo tìm ra nghiệm tối ưu
nhưng thường tốn kém thời gian, heuristic tiếp cận theo hướng ``thực
dụng'' -- chấp nhận lời giải tốt đủ nhanh trong những bài toán phức tạp.

Cơ sở lý thuyết quan trọng của heuristic là nguyên tắc phỏng đoán (rule
of thumb). Thay vì duyệt toàn bộ không gian tìm kiếm, heuristic sử dụng
các quy tắc hoặc chiến lược để ưu tiên những hướng đi có khả năng dẫn
đến lời giải tốt. Điều này giúp giảm đáng kể chi phí tính toán, đặc biệt
trong các bài toán NP-hard (như tối ưu hóa, tìm đường, lập lịch).

Heuristic cũng phổ biến khái niệm tìm kiếm gần đúng (approximate
search). Ví dụ, trong bài toán tìm đường, thay vì duyệt toàn bộ đồ thị,
heuristic có thể dựa vào khoảng cách ước lượng (như Euclidean distance)
để hướng dẫn quá trình tìm kiếm.

Do không gian tìm kiếm phương án tối ưu là quá lớn, em sử dụng phương
pháp Heuristic (dựa trên kinh nghiệm) thông qua thuật toán tham lam
(Greedy).

Tại mỗi bước (mỗi khung giờ), thuật toán sẽ chọn ngẫu nhiên một ứng viên
tốt nhất thỏa mãn các ràng buộc hiện tại mà không cần quay lui
(Backtracking). Cách tiếp cận này tuy đơn giản nhưng đảm bảo tốc độ phản
hồi nhanh và kết quả luôn khả thi.

\newpage
\section{CHƯƠNG 2: PHÂN TÍCH VÀ THIẾT KẾ
WEB}\label{chux1b0ux1a1ng-2-phuxe2n-tuxedch-vuxe0-thiux1ebft-kux1ebf-web}

\begin{enumerate}
\def\labelenumi{\arabic{enumi}.}
\item
  \protect\phantomsection\label{_Toc215924007}{}\textbf{Phân tích Tác
  nhân hệ thống (System Actors Analysis)}
\end{enumerate}

Trong kỹ nghệ phần mềm, bước đầu tiên của thiết kế hệ thống là xác định
chính xác "ai" sẽ tương tác với phần mềm. Đối với hệ thống Dana Travel,
việc phân tích tác nhân không chỉ dừng lại ở việc định danh
(Guest/Admin) mà phải đi sâu vào đặc điểm hành vi để thiết kế cơ chế bảo
mật và trải nghiệm tương ứng. Đặc tả yêu cầu hệ thống

\begin{enumerate}
\def\labelenumi{\arabic{enumi}.}
\item
  \protect\phantomsection\label{_Toc215924008}{}\textbf{Tác nhân: Khách
  du lịch vãng lai (Guest User)}
\end{enumerate}

Đây là đối tượng sử dụng chính, chiếm đa số lưu lượng truy cập của hệ
thống.

\begin{itemize}
\item
  Đặc điểm kỹ thuật:

  \begin{itemize}
  \item
    Trạng thái phiên (Session State): Stateless (Phi trạng thái). Hệ
    thống không duy trì kết nối bền vững (Persistent Connection) với
    Guest. Mọi dữ liệu về phiên làm việc của họ (lịch trình đang tạo dở,
    lịch sử chat) được lưu trữ hoàn toàn tại phía Client (Trình duyệt)
    thông qua localStorage hoặc sessionStorage.
  \item
    Quyền hạn truy cập (Access Control): Guest có quyền đọc (READ) dữ
    liệu công khai từ API /api/locations và quyền ghi (WRITE) hạn chế
    vào các endpoint xử lý logic như /api/itinerary/generate hay
    /api/chat. Họ tuyệt đối không có quyền truy cập vào các API quản trị
    (/api/admin/*).
  \end{itemize}
\item
  Hành vi tương tác (Interaction Behavior):

  \begin{itemize}
  \item
    Guest thường có xu hướng thoát trang ngay lập tức nếu gặp rào cản
    đăng nhập. Do đó, quyết định thiết kế quan trọng ở đây là
    "Anonymous-First" (Ẩn danh trước tiên). Hệ thống chấp nhận đánh đổi
    việc không thu thập được email người dùng để đổi lấy tỷ lệ chuyển
    đổi (Conversion Rate) cao hơn cho chức năng tạo lịch trình.

    \begin{enumerate}
    \def\labelenumi{\arabic{enumi}.}
    \item
      \protect\phantomsection\label{_Toc215924009}{}\textbf{Tác nhân:
      Quản trị viên hệ thống (System Administrator)}
    \end{enumerate}
  \end{itemize}
\end{itemize}

Đây là nhóm tác nhân nội bộ, chịu trách nhiệm duy trì tính đúng đắn của
dữ liệu.

\begin{itemize}
\item
  Đặc điểm kỹ thuật:

  \begin{itemize}
  \item
    Cơ chế xác thực (Authentication): Bắt buộc phải thông qua quy trình
    đăng nhập bảo mật. Hệ thống sử dụng cơ chế Token-based (JWT hoặc
    Session Cookie) để định danh Admin trong mỗi request gửi lên Server.
  \item
    Phân quyền (Authorization): Admin có toàn quyền (FULL ACCESS) trên
    các bảng dữ liệu Location, Knowledge, SystemConfig.
  \end{itemize}
\item
  Trách nhiệm vận hành:

  \begin{itemize}
  \item
    Admin không chỉ nhập liệu mà còn đóng vai trò "Huấn luyện viên AI".
    Thông qua việc cập nhật bảng Knowledge, Admin gián tiếp điều chỉnh
    câu trả lời của Chatbot mà không cần can thiệp vào mã nguồn
    (Codebase).
  \end{itemize}

  \begin{enumerate}
  \def\labelenumi{\arabic{enumi}.}
  \item
    \protect\phantomsection\label{_Toc215924010}{}\textbf{Xây dựng Chân
    dung người dùng (User Personas \& Algorithm Mapping)}
  \end{enumerate}
\end{itemize}

Để thuật toán xếp lịch hoạt động hiệu quả, hệ thống không thể đối xử với
mọi người dùng như nhau. Em đã mô hình hóa 3 chân dung người dùng điển
hình. Các chân dung này không chỉ là lý thuyết marketing mà được ánh xạ
trực tiếp vào các tham số lọc của thuật toán generateDayScheduleStrict.

\begin{enumerate}
\def\labelenumi{\arabic{enumi}.}
\item
  \protect\phantomsection\label{_Toc215924011}{}\textbf{Persona A: "Sinh
  viên du lịch bụi" (The Budget Backpacker)}
\end{enumerate}

\begin{itemize}
\item
  Mô tả: Nhóm khách trẻ, sinh viên, đi một mình hoặc nhóm bạn nhỏ.
\item
  Tham số đầu vào hệ thống:

  \begin{itemize}
  \item
    tripType: "SOLO" hoặc "FRIENDS".
  \item
    budget: "BUDGET" (Thấp).
  \end{itemize}
\item
  Ánh xạ vào logic thuật toán (Algorithm Mapping):

  \begin{itemize}
  \item
    Bộ lọc giá (Price Filter): Khi phát hiện Persona này, thuật toán
    kích hoạt bộ lọc nghiêm ngặt priceLevel ===
    \textquotesingle BUDGET\textquotesingle{} (\textless{} 100.000 VNĐ).
    Các địa điểm có priceLevel ===
    \textquotesingle LUXURY\textquotesingle{} sẽ bị loại bỏ hoàn toàn
    khỏi không gian tìm kiếm (Search Space).
  \item
    Ưu tiên loại hình: Thuật toán tăng trọng số (Weight) cho các địa
    điểm có tag street\_food (ẩm thực đường phố) và các địa điểm tham
    quan miễn phí (Cầu Rồng, Biển Mỹ Khê).
  \item
    Chấp nhận rủi ro: Persona này chấp nhận di chuyển xa để tiết kiệm
    chi phí, do đó bán kính tìm kiếm địa điểm (nếu có tính năng bản đồ)
    sẽ được mở rộng tối đa.

    \begin{enumerate}
    \def\labelenumi{\arabic{enumi}.}
    \item
      \protect\phantomsection\label{_Toc215924012}{}\textbf{Persona B:
      "Gia đình nghỉ dưỡng" (The Family Vacationer)}
    \end{enumerate}
  \end{itemize}
\end{itemize}

\begin{itemize}
\item
  Mô tả: Gia đình có trẻ nhỏ hoặc người lớn tuổi, chú trọng sự an toàn
  và tiện nghi.
\item
  Tham số đầu vào hệ thống:

  \begin{itemize}
  \item
    tripType: "FAMILY".
  \item
    budget: "MODERATE" hoặc "LUXURY".
  \end{itemize}
\item
  Ánh xạ vào logic thuật toán:

  \begin{itemize}
  \item
    Bộ lọc an toàn (Safety Filter): Thuật toán sẽ kiểm tra thẻ tag
    family\_friendly. Các địa điểm có tính chất mạo hiểm (leo núi,
    trekking) hoặc ồn ào (bar, pub) sẽ bị loại bỏ.
  \item
    Mật độ lịch trình (Density Control): Đối với nhóm FAMILY, thuật toán
    giới hạn số lượng địa điểm tối đa trong một buổi là 2 địa điểm (thay
    vì 3-4 như nhóm sinh viên) để đảm bảo sức khỏe cho thành viên đoàn.
  \item
    Ưu tiên tiện ích: Ưu tiên chọn các nhà hàng có không gian rộng, có
    ghế trẻ em (dựa trên mô tả trong DB).

    \begin{enumerate}
    \def\labelenumi{\arabic{enumi}.}
    \item
      \protect\phantomsection\label{_Toc215924013}{}\textbf{Persona C:
      "Cặp đôi lãng mạn" (The Honeymooners)}
    \end{enumerate}
  \end{itemize}
\end{itemize}

\begin{itemize}
\item
  Mô tả: Các cặp đôi đi trăng mật hoặc kỷ niệm, sẵn sàng chi trả cho
  trải nghiệm cảm xúc.
\item
  Tham số đầu vào hệ thống:

  \begin{itemize}
  \item
    tripType: "COUPLE".
  \item
    budget: "MODERATE" hoặc "LUXURY".
  \end{itemize}
\item
  Ánh xạ vào logic thuật toán:

  \begin{itemize}
  \item
    Bộ lọc cảm xúc (Vibe Filter): Thuật toán ưu tiên tìm kiếm các từ
    khóa romantic, view\_dep, private trong trường tags hoặc description
    của địa điểm.
  \item
    Sắp xếp thời gian: Bữa tối (Dinner) của nhóm này sẽ luôn được ưu
    tiên chọn các nhà hàng ven sông hoặc ven biển, và khung giờ buổi tối
    sẽ gợi ý các hoạt động nhẹ nhàng như "Du thuyền sông Hàn" hoặc "Cafe
    Acoustic".
  \end{itemize}
\end{itemize}

Việc phân định rạch ròi các Persona này giúp hệ thống biến một input đơn
giản (Loại nhóm) thành một tập hợp các quy tắc logic phức tạp, từ đó
sinh ra các lịch trình mang tính cá nhân hóa cao (Personalization),
tránh tình trạng "râu ông nọ cắm cằm bà kia".

\begin{enumerate}
\def\labelenumi{\arabic{enumi}.}
\item
  \protect\phantomsection\label{_Toc215924014}{}\textbf{Đặc tả chi tiết
  yêu cầu chức năng} \textbf{: Lập lịch trình tự động (Automated
  Itinerary Generation)}
\end{enumerate}

Đây là chức năng cốt lõi (Core Feature), định hình giá trị của sản phẩm.

\begin{enumerate}
\def\labelenumi{\arabic{enumi}.}
\item
  \protect\phantomsection\label{_Toc215924015}{}\textbf{Dữ liệu đầu vào
  (Input)}
\end{enumerate}

Người dùng cung cấp một tập hợp các tham số ràng buộc (C) thông qua Form
giao diện:

\begin{itemize}
\item
  D (Duration): Số nguyên, \(1 \leq D \leq 7\). (Giới hạn 7 ngày để đảm
  bảo hiệu năng tính toán).
\item
  T (Trip Type): Enum \{SOLO, COUPLE, FAMILY, FRIENDS\}.
\item
  B (Budget): Enum \{BUDGET, MODERATE, LUXURY\}.
\item
  I (Interests - Tùy chọn): Mảng String \{culture, food, nature, ...\}.

  \begin{enumerate}
  \def\labelenumi{\arabic{enumi}.}
  \item
    \protect\phantomsection\label{_Toc215924016}{}\textbf{Quy trình xử
    lý (Process)}
  \end{enumerate}
\end{itemize}

Hệ thống thực hiện quy trình xử lý tuần tự qua 4 bước tại Backend

Bước 1 - Tiền xử lý (Pre-processing): Controller nhận Request, validate
dữ liệu (đảm bảo số ngày không âm, budget hợp lệ).

Bước 2 - Truy vấn dữ liệu thô (Raw Data Retrieval): Service gọi xuống
Database (Prisma) để lấy toàn bộ danh sách địa điểm đang hoạt động
(status = \textquotesingle ACTIVE\textquotesingle). Lưu ý: Tại bước này
chưa lọc kỹ, chỉ lấy dữ liệu thô lên RAM.

Bước 3 - Lọc và Sắp xếp (Filtering \& Scheduling): Đây là bước chạy
thuật toán generateDayScheduleStrict.

\begin{itemize}
\item
  Hệ thống khởi tạo một mảng rỗng ItineraryResult.
\item
  Vòng lặp chạy từ Day 1 đến Day D.
\item
  Trong mỗi ngày, chạy tiếp vòng lặp qua 7 pha (Sáng/Trưa/Chiều/Tối...).
  Tại mỗi pha, áp dụng các luật logic (như đã phân tích ở Persona) để
  chọn ra ứng viên tốt nhất.
\item
  Cơ chế "Backtracking" nhẹ: Nếu một ngày không tìm được địa điểm ăn
  trưa, hệ thống sẽ thử nới lỏng điều kiện ngân sách để tìm lại.
\end{itemize}

Bước 4 - Đóng gói (Packaging): Kết quả được format lại theo cấu trúc
JSON chuẩn, kèm theo các thông tin bổ trợ (tổng chi phí ước tính, tổng
số địa điểm).

\begin{enumerate}
\def\labelenumi{\arabic{enumi}.}
\item
  \protect\phantomsection\label{_Toc215924017}{}\textbf{Dữ liệu đầu ra
  (Output)}
\end{enumerate}

Một đối tượng JSON có cấu trúc phân cấp:

\{

~ "tripDetails": \{ "duration": 3, "type": "COUPLE" \},

~ "schedule": {[}

~ ~ \{

~ ~ ~ "day": 1,

~ ~ ~ "activities": {[}

~ ~ ~ ~ \{ "time": "08:00", "place": "Biển Mỹ Khê", "type": "BEACH" \},

~ ~ ~ ~ \{ "time": "12:00", "place": "Bánh tráng cuốn thịt heo", "type":
"FOOD" \}

~ ~ ~ {]}

~ ~ \},

~ ~ ...

~ {]}

\}

\begin{enumerate}
\def\labelenumi{\arabic{enumi}.}
\item
  \protect\phantomsection\label{_Toc215924018}{}\textbf{Đặc tả chức
  năng: Trợ lý ảo tư vấn (AI Chatbot Consultation)}
\end{enumerate}

Chức năng này giải quyết nhu cầu tra cứu thông tin phi cấu trúc mà thuật
toán lập lịch không làm được.

\begin{enumerate}
\def\labelenumi{\arabic{enumi}.}
\item
  \protect\phantomsection\label{_Toc215924019}{}\textbf{Dữ liệu đầu vào
  (Input)}
\end{enumerate}

\begin{itemize}
\item
  message: Chuỗi văn bản tự do của người dùng (Ví dụ: "Hôm nay trời mưa
  thì đi đâu?").
\item
  history: Mảng các tin nhắn trước đó (để AI hiểu ngữ cảnh hội thoại).

  \begin{enumerate}
  \def\labelenumi{\arabic{enumi}.}
  \item
    \protect\phantomsection\label{_Toc215924020}{}\textbf{Quy trình xử
    lý (Process) Quy trình áp dụng kỹ thuật RAG (Retrieval Augmented
    Generation):}
  \end{enumerate}
\end{itemize}

\begin{enumerate}
\def\labelenumi{\arabic{enumi}.}
\item
  Phân tích ý định (Intent Recognition): Hệ thống quét từ khóa trong câu
  hỏi. Nếu phát hiện từ khóa liên quan đến địa điểm (ví dụ: "Bà Nà",
  "Hội An"), hệ thống kích hoạt chế độ "Tra cứu dữ liệu".
\item
  Truy xuất ngữ cảnh (Context Retrieval): Hệ thống thực hiện câu lệnh
  truy vấn LIKE hoặc Full-text Search vào bảng Location và Knowledge.
  Mục tiêu là tìm ra các đoạn văn bản sự thật (Ground Truth). Ví dụ: Tìm
  thấy bản ghi "Bà Nà Hills - Giá vé 900k - Mở cửa 7h-22h".
\item
  Kỹ thuật Prompt Engineering: Hệ thống ghép câu hỏi của người dùng và
  dữ liệu vừa tìm được vào một khuôn mẫu (Template):
\end{enumerate}

\begin{quote}
"Bạn là trợ lý du lịch Dana Travel. Dữ liệu hệ thống cho biết: {[}Bà Nà
Hills giá 900k...{]}. Hãy trả lời câu hỏi: {[}Vé Bà Nà bao nhiêu?{]} dựa
trên dữ liệu trên."
\end{quote}

\begin{enumerate}
\def\labelenumi{\arabic{enumi}.}
\setcounter{enumi}{3}
\item
  Gọi API (LLM Inference): Gửi Prompt đã ghép sang Google Gemini API và
  chờ phản hồi.
\item
  Hậu xử lý (Post-processing): Nhận văn bản trả về, lọc bỏ các ký tự lạ
  (nếu có) và lưu vào lịch sử chat.

  \begin{enumerate}
  \def\labelenumii{\arabic{enumii}.}
  \item
    \protect\phantomsection\label{_Toc215924021}{}\textbf{Dữ liệu đầu ra
    (Output)}
  \end{enumerate}
\end{enumerate}

Chuỗi văn bản trả lời tự nhiên, chính xác, có trích dẫn số liệu từ
Database (nếu có)\textbf{\hfill\break
}

\begin{enumerate}
\def\labelenumi{\arabic{enumi}.}
\item
  \protect\phantomsection\label{_Toc215924022}{}\textbf{Đặc tả chức
  năng: Quản trị dữ liệu (Data Management)}
\end{enumerate}

Chức năng dành riêng cho Admin để duy trì hệ thống.

\begin{enumerate}
\def\labelenumi{\arabic{enumi}.}
\item
  \protect\phantomsection\label{_Toc215924023}{}\textbf{Quản lý Địa điểm
  (CRUD Locations)}
\end{enumerate}

\begin{itemize}
\item
  Input: Form nhập liệu gồm Tên, Mô tả, Giá, Giờ mở cửa, Ảnh, Tọa độ.
\item
  Process:
\end{itemize}

\begin{itemize}
\item
  Validate dữ liệu: Giờ mở cửa phải nhỏ hơn Giờ đóng cửa, Giá tiền không
  được âm.
\item
  Lưu vào Database thông qua Prisma create hoặc update.
\end{itemize}

\begin{itemize}
\item
  Output: Thông báo thành công/thất bại và cập nhật danh sách hiển thị
  ngay lập tức.

  \begin{enumerate}
  \def\labelenumi{\arabic{enumi}.}
  \item
    \protect\phantomsection\label{_Toc215924024}{}\textbf{Quản lý Tri
    thức (Knowledge Base Management)}
  \end{enumerate}
\end{itemize}

\begin{itemize}
\item
  Mục tiêu: Cung cấp "não bộ" cho AI.
\item
  Cơ chế: Admin nhập các đoạn văn bản (Text Chunks) về các chủ đề chung
  (Thời tiết, Mẹo vặt, Văn hóa). Khi AI gặp câu hỏi không tìm thấy địa
  điểm cụ thể, nó sẽ dùng dữ liệu từ bảng này để trả lời. Ví dụ: Dữ liệu
  "Tháng 10-12 là mùa mưa bão" sẽ giúp AI trả lời câu hỏi "Tháng 11 đi
  Đà Nẵng có ổn không?".

  \begin{enumerate}
  \def\labelenumi{\arabic{enumi}.}
  \item
    \protect\phantomsection\label{_Toc215924025}{}\textbf{Thống kê và
    Giám sát (Monitoring)}
  \end{enumerate}
\end{itemize}

\begin{itemize}
\item
  Hệ thống ghi lại nhật ký (AccessLog) mỗi khi có người dùng tạo lịch
  trình hoặc chat.
\item
  Admin xem được các chỉ số: Số lượng lịch trình được tạo trong ngày,
  Top địa điểm được gợi ý nhiều nhất. Điều này giúp Admin biết xu hướng
  du lịch để cập nhật dữ liệu (ví dụ: thấy người dùng hay hỏi về "Cắm
  trại", Admin sẽ thêm các địa điểm Camping vào DB).

  \begin{enumerate}
  \def\labelenumi{\arabic{enumi}.}
  \item
    \protect\phantomsection\label{_Toc215924026}{}\textbf{Hiệu năng
    (Performance Efficiency)}

    \begin{enumerate}
    \def\labelenumii{\arabic{enumii}.}
    \item
      \protect\phantomsection\label{_Toc215924027}{}\textbf{Thời gian
      phản hồi (Response Time Latency)}
    \end{enumerate}
  \end{enumerate}
\end{itemize}

\begin{itemize}
\item
  Chỉ tiêu: Hệ thống phải đảm bảo thời gian phản hồi cho chức năng sinh
  lịch trình (POST /api/itinerary/generate) là dưới 2000ms (2 giây) đối
  với lịch trình 7 ngày.
\item
  Phân tích kỹ thuật: Thuật toán sinh lịch trình hiện tại có độ phức tạp
  tính toán là \$O(D \textbackslash times P \textbackslash times L)\$,
  trong đó \$D\$ là số ngày (max 7), \$P\$ là số pha trong ngày (cố định
  7), và \$L\$ là số lượng địa điểm trong Database
  (\textasciitilde100-200). Với không gian mẫu nhỏ này, việc xử lý trên
  Node.js (V8 Engine) chỉ mất khoảng 50-100ms cho việc tính toán CPU.
\item
  Yếu tố độ trễ: Phần lớn thời gian trễ sẽ đến từ Network Latency (đường
  truyền mạng) và Database Query. Để tối ưu, hệ thống sử dụng cơ chế
  Eager Loading (tải dữ liệu một lần) thay vì Lazy Loading (tải lắt
  nhắt) khi truy vấn danh sách địa điểm từ SQLite.

  \begin{enumerate}
  \def\labelenumi{\arabic{enumi}.}
  \item
    \protect\phantomsection\label{_Toc215924028}{}\textbf{Khả năng chịu
    tải (Throughput)}
  \end{enumerate}
\end{itemize}

\begin{itemize}
\item
  Chỉ tiêu: Hệ thống hỗ trợ tối thiểu 50 Concurrent Users (người dùng
  đồng thời) trên cấu hình phần cứng cơ bản (1 vCPU, 1GB RAM).
\item
  Giải pháp: Kiến trúc Node.js đơn luồng (Single-thread) với
  Non-blocking I/O cho phép xử lý hàng nghìn kết nối đồng thời mà không
  bị tắc nghẽn (Blocking). Các tác vụ nặng như gọi API Google Gemini
  được xử lý bất đồng bộ (Asynchronous), giải phóng Event Loop để tiếp
  nhận request mới ngay lập tức.

  \begin{enumerate}
  \def\labelenumi{\arabic{enumi}.}
  \item
    \protect\phantomsection\label{_Toc215924029}{}\textbf{Tính tin cậy
    và Sẵn sàng (Reliability \& Availability)}

    \begin{enumerate}
    \def\labelenumii{\arabic{enumii}.}
    \item
      \protect\phantomsection\label{_Toc215924030}{}\textbf{Cơ chế hoạt
      động độc lập (Offline Capability)}
    \end{enumerate}
  \end{enumerate}
\end{itemize}

Mặc dù là Web App, hệ thống được thiết kế để giảm thiểu sự phụ thuộc vào
dịch vụ bên thứ 3. Nếu API Google Gemini bị lỗi (Downtime) hoặc hết
quota, chức năng Chatbot sẽ trả về thông báo lỗi thân thiện ("Bot đang
bận"), nhưng chức năng cốt lõi là Sinh lịch trình vẫn hoạt động bình
thường 100\% do chạy hoàn toàn bằng logic nội bộ (Rule-based Algorithm)
trên Server.

\begin{enumerate}
\def\labelenumi{\arabic{enumi}.}
\setcounter{enumi}{1}
\item
  \protect\phantomsection\label{_Toc215924031}{}\textbf{Xử lý lỗi toàn
  cục (Global Error Handling)}
\end{enumerate}

\begin{itemize}
\item
  Hệ thống áp dụng mẫu thiết kế (Design Pattern) "Centralized Error
  Handling". Tất cả các ngoại lệ (Exceptions) không được bắt (Uncaught)
  trong Controller hay Service đều được chuyển về một Middleware duy
  nhất là error.handler.middleware.js.
\item
  Lợi ích: Đảm bảo Server không bao giờ bị "Crash" (sập) do một lỗi
  logic nhỏ, đồng thời chuẩn hóa định dạng lỗi trả về cho Frontend (luôn
  là JSON \{ success: false, message: ... \}), giúp giao diện không bị
  trắng trang (White Screen of Death).

  \begin{enumerate}
  \def\labelenumi{\arabic{enumi}.}
  \item
    \protect\phantomsection\label{_Toc215924032}{}\textbf{An toàn và Bảo
    mật thông tin (Security)}

    \begin{enumerate}
    \def\labelenumii{\arabic{enumii}.}
    \item
      \protect\phantomsection\label{_Toc215924033}{}\textbf{Kiểm soát
      truy cập (Access Control)}
    \end{enumerate}
  \end{enumerate}
\end{itemize}

\begin{itemize}
\item
  Phía Guest: Áp dụng nguyên lý "Quyền tối thiểu" (Least Privilege). API
  dành cho Guest chỉ cho phép phương thức GET trên tài nguyên công khai.
  Các phương thức POST bị giới hạn rate-limit (ví dụ: 10 request/phút)
  để chống tấn công DDoS.
\item
  Phía Admin: Toàn bộ các route bắt đầu bằng /api/admin/* được bảo vệ
  bởi middleware adminAuth. Middleware này kiểm tra sự tồn tại và tính
  hợp lệ của Session Token trong Header của request. Bất kỳ truy cập nào
  không có Token sẽ bị từ chối với mã lỗi HTTP 401 (Unauthorized).

  \begin{enumerate}
  \def\labelenumi{\arabic{enumi}.}
  \item
    \protect\phantomsection\label{_Toc215924034}{}\textbf{Bảo mật dữ
    liệu nhạy cảm}
  \end{enumerate}
\end{itemize}

\begin{itemize}
\item
  Mật khẩu của Admin không bao giờ được lưu dưới dạng văn bản thuần
  (Plain Text). Hệ thống sử dụng thư viện bcrypt để băm (Hash) mật khẩu
  kèm theo Salt (muối - chuỗi ngẫu nhiên) trước khi lưu vào Database.
  Điều này đảm bảo ngay cả khi file Database bị đánh cắp, hacker cũng
  không thể dịch ngược ra mật khẩu gốc.
\item
  Hệ thống sử dụng cơ chế CORS (Cross-Origin Resource Sharing) để chỉ
  cho phép các tên miền (Domain) tin cậy được gọi API, ngăn chặn các
  trang web độc hại khác mạo danh người dùng gửi request.

  \begin{enumerate}
  \def\labelenumi{\arabic{enumi}.}
  \item
    \protect\phantomsection\label{_Toc215924035}{}\textbf{Khả năng bảo
    trì và Mở rộng (Maintainability \& Scalability)}

    \begin{enumerate}
    \def\labelenumii{\arabic{enumii}.}
    \item
      \protect\phantomsection\label{_Toc215924036}{}\textbf{Kiến trúc
      Modular Monolith}
    \end{enumerate}
  \end{enumerate}
\end{itemize}

Mã nguồn được tổ chức theo module (Chat, Itinerary, Location). Mỗi
module có đầy đủ các tầng Controller-Service-Route riêng biệt. Điều này
giúp việc bảo trì dễ dàng: khi muốn sửa lỗi Chatbot, lập trình viên chỉ
cần thao tác trong thư mục src/services/chatbot.service.js mà không sợ
ảnh hưởng đến logic tạo lịch trình.

\begin{enumerate}
\def\labelenumi{\arabic{enumi}.}
\setcounter{enumi}{1}
\item
  \protect\phantomsection\label{_Toc215924037}{}\textbf{Khả năng mở rộng
  Database}
\end{enumerate}

Hiện tại hệ thống dùng SQLite cho sự tiện lợi. Tuy nhiên, nhờ sử dụng
Prisma ORM làm lớp trừu tượng (Abstraction Layer), việc chuyển đổi sang
MySQL hay PostgreSQL trong tương lai (khi lượng người dùng tăng lên hàng
triệu) chỉ đơn giản là thay đổi chuỗi kết nối (Connection String) trong
file .env mà không cần viết lại bất kỳ dòng code logic nào.

\begin{enumerate}
\def\labelenumi{\arabic{enumi}.}
\setcounter{enumi}{1}
\item
  \protect\phantomsection\label{_Toc215924038}{}\textbf{Quy trình nghiệp
  vụ Sinh lịch trình du lịch (Itinerary Generation Process)}
\end{enumerate}

Đây là quy trình phức tạp nhất, nơi "trí tuệ" của hệ thống được thể
hiện. Quy trình này biến đổi dữ liệu đầu vào thô sơ thành một kế hoạch
chi tiết.

\begin{itemize}
\item
  B1: Tiếp nhận và Validate (Input Validation Layer)

  \begin{itemize}
  \item
    Người dùng gửi form \{ duration: 3, budget: "MODERATE" \}.
  \item
    Logic kiểm tra:

    \begin{itemize}
    \item
      Nếu duration \textless{} 1 hoặc duration \textgreater{} 7
      -\textgreater{} Trả lỗi 400 (Bad Request).
    \item
      Nếu budget không thuộc tập \{BUDGET, MODERATE, LUXURY\}
      -\textgreater{} Gán mặc định là MODERATE.
    \end{itemize}
  \end{itemize}
\item
  B2: Khởi tạo không gian tìm kiếm (Search Space Initialization)

  \begin{itemize}
  \item
    Hệ thống tải toàn bộ địa điểm có trạng thái ACTIVE vào bộ nhớ.
  \item
    Phân loại dữ liệu: Dữ liệu được chia thành các "Bucket" (xô) nhỏ hơn
    để tối ưu tốc độ truy xuất:

    \begin{itemize}
    \item
      foodLocations: Danh sách quán ăn.
    \item
      attractionLocations: Danh sách điểm tham quan.
    \item
      beachLocations: Danh sách bãi biển.
    \item
      coffeeLocations: Danh sách quán cafe/bar.
    \end{itemize}
  \end{itemize}
\item
  B3: Vòng lặp Heuristic (The Heuristic Loop)

  \begin{itemize}
  \item
    Quy trình chạy vòng lặp for (day = 1 to duration). Trong mỗi ngày,
    hệ thống thực hiện tuần tự 7 bước gán (Assignment Steps).
  \item
    Cơ chế Backtracking cục bộ: Tại mỗi bước (ví dụ: Chọn quán ăn trưa),
    hệ thống thử chọn một địa điểm ngẫu nhiên trong "Bucket" tương ứng.

    \begin{itemize}
    \item
      Kiểm tra 1: Quán này đã đi chưa? (Check visitedSet).
    \item
      Kiểm tra 2: Quán có mở cửa lúc 12:00 không? (Check openingTime).
    \item
      Kiểm tra 3: Giá quán có phù hợp Budget không? (Check priceLevel).
    \end{itemize}
  \item
    Nếu cả 3 kiểm tra đều PASS -\textgreater{} Thêm vào lịch trình
    -\textgreater{} Đánh dấu đã đi.
  \item
    Nếu FAIL -\textgreater{} Thử quán khác. Nếu thử hết danh sách vẫn
    không được -\textgreater{} Kích hoạt cơ chế Relaxation (Nới lỏng
    ràng buộc giá) để chọn đại một quán bất kỳ (đảm bảo không bị null).
  \end{itemize}
\item
  B4: Hậu xử lý và Định dạng (Post-processing)

  \begin{itemize}
  \item
    Tính toán tổng chi phí ước tính = Tổng averagePrice của tất cả địa
    điểm đã chọn.
  \item
    Gắn thêm các thông tin bổ trợ như hình ảnh, mô tả ngắn vào JSON kết
    quả.
  \end{itemize}

  \begin{enumerate}
  \def\labelenumi{\arabic{enumi}.}
  \item
    \protect\phantomsection\label{_Toc215924039}{}\textbf{Quy trình
    nghiệp vụ: Hội thoại thông minh (Conversation Flow)}
  \end{enumerate}
\end{itemize}

\begin{itemize}
\item
  Luồng dữ liệu (Data Flow Diagram - Level 1):
\end{itemize}

\includegraphics[width=3.85591in,height=3.38894in,alt={A diagram of a child\textquotesingle s development AI-generated content may be incorrect.}]{media/image17.png}

{\def\LTcaptype{none} % do not increment counter
\begin{longtable}[]{@{}
  >{\raggedright\arraybackslash}p{(\linewidth - 4\tabcolsep) * \real{0.2102}}
  >{\raggedright\arraybackslash}p{(\linewidth - 4\tabcolsep) * \real{0.4598}}
  >{\raggedright\arraybackslash}p{(\linewidth - 4\tabcolsep) * \real{0.3300}}@{}}
\toprule\noalign{}
\begin{minipage}[b]{\linewidth}\centering
\textbf{Pha}
\end{minipage} & \begin{minipage}[b]{\linewidth}\centering
\textbf{Mô Tả}
\end{minipage} & \begin{minipage}[b]{\linewidth}\centering
\textbf{Dữ Liệu Chuyển}
\end{minipage} \\
\midrule\noalign{}
\endhead
\bottomrule\noalign{}
\endlastfoot
\textbf{Tạo Lịch Trình (1-10)} & Người dùng nhập yêu cầu → Frontend →
API → Service → Database → Tính Toán → Trả Kết Quả & Form JSON, Danh
sách địa điểm, Lịch trình, Khoảng cách \\
\textbf{Tương Tác Chat (11-20)} & Người dùng gửi tin nhắn → API →
Service → Gemini → Lưu Log → Trả Phản Hồi & Tin nhắn văn bản, Phản hồi
AI, Log chat \\
\textbf{Lưu Trữ} & Dữ liệu tạm thời lưu trên máy khách, dữ liệu quan
trọng lưu DB & localStorage, PostgreSQL/SQLite \\
\end{longtable}
}

\begin{itemize}
\item
  \textbf{Điểm kiểm soát logic (Logic Gate):}

  \begin{itemize}
  \item
    Trong quá trình này, có một điểm kiểm soát quan trọng tại bước 3.
    Nếu Database Query trả về rỗng (tức là người dùng hỏi về một địa
    điểm lạ hoắc không có trong hệ thống, ví dụ "Tháp Eiffel"), Service
    sẽ chuyển hướng sang truy vấn bảng Knowledge (Tri thức chung). Nếu
    bảng Knowledge cũng không có, Service sẽ chèn một chỉ thị vào
    Prompt: "Dữ liệu không có thông tin này, hãy trả lời khéo léo là bạn
    chưa được cập nhật dữ liệu về địa điểm đó". Điều này ngăn chặn việc
    AI tự bịa ra thông tin sai sự thật về một địa điểm không tồn tại.
  \end{itemize}

  \begin{enumerate}
  \def\labelenumi{\arabic{enumi}.}
  \item
    \protect\phantomsection\label{_Toc215924040}{}\textbf{Sơ đồ trạng
    thái của Địa điểm (Entity State Diagram)}
  \end{enumerate}
\end{itemize}

\includegraphics[width=5.20471in,height=3.48364in,alt={A diagram of a diagram AI-generated content may be incorrect.}]{media/image18.png}\textbf{\hfill\break
}

\textbf{Bảng chuyển đổi trạng thái chi tiết:}

{\def\LTcaptype{none} % do not increment counter
\begin{longtable}[]{@{}
  >{\raggedright\arraybackslash}p{(\linewidth - 8\tabcolsep) * \real{0.1925}}
  >{\raggedright\arraybackslash}p{(\linewidth - 8\tabcolsep) * \real{0.2294}}
  >{\raggedright\arraybackslash}p{(\linewidth - 8\tabcolsep) * \real{0.2044}}
  >{\raggedright\arraybackslash}p{(\linewidth - 8\tabcolsep) * \real{0.1597}}
  >{\raggedright\arraybackslash}p{(\linewidth - 8\tabcolsep) * \real{0.2141}}@{}}
\toprule\noalign{}
\begin{minipage}[b]{\linewidth}\raggedright
\textbf{Trạng Thái Hiện Tại}
\end{minipage} & \begin{minipage}[b]{\linewidth}\raggedright
\textbf{Sự Kiện}
\end{minipage} & \begin{minipage}[b]{\linewidth}\raggedright
\textbf{Trạng Thái Tiếp Theo}
\end{minipage} & \begin{minipage}[b]{\linewidth}\raggedright
\textbf{Điều Kiện}
\end{minipage} & \begin{minipage}[b]{\linewidth}\raggedright
\textbf{Ghi Chú}
\end{minipage} \\
\midrule\noalign{}
\endhead
\bottomrule\noalign{}
\endlastfoot
Chưa Kich Hoạt & Admin bật (active=true) & KichHoạt & - & Địa điểm sẵn
sàng sử dụng \\
Kich Hoạt & Admin chỉnh sửa & ChỉnhSửa & - & Đang cập nhật thông tin \\
Chỉnh Sửa & Lưu hoàn thành & KichHoạt & Data hợp lệ & Quay về trạng thái
chính \\
Kich Hoạt & Tạm dừng hoạt động & TamDừng & Có lý do & Sẽ quay lại \\
Tam Dừng & Quay lại hoạt động & KichHoạt & - & Xác nhận không còn sự
cố \\
Bất Kỳ & Xoá logic & XóaMềm & Quyết định Admin & deleted\_at được ghi
lại \\
XóaMềm & Khôi phục & Trạng thái cũ & Admin yêu cầu & deleted\_at =
null \\
\end{longtable}
}

\textbf{\hfill\break
}

\begin{enumerate}
\def\labelenumi{\arabic{enumi}.}
\setcounter{enumi}{1}
\item
  \protect\phantomsection\label{_Toc215924041}{}\textbf{Kiến trúc tổng
  thể (High-Level Architecture)}
\end{enumerate}

Hệ thống được xây dựng dựa trên mô hình Client-Server truyền thống nhưng
được hiện đại hóa bằng các công nghệ mới nhất.

\includegraphics[width=6.53125in,height=5.34444in,alt={A diagram of a company AI-generated content may be incorrect.}]{media/image19.png}

\begin{itemize}
\item
  Tầng Client (Frontend):

  \begin{itemize}
  \item
    Được xây dựng dưới dạng Single Page Application (SPA) sử dụng thư
    viện ReactJS.
  \item
    Tầng này chịu trách nhiệm hiển thị giao diện (Presentation) và quản
    lý trạng thái phiên làm việc (Session State).
  \item
    Mã nguồn Frontend được biên dịch và đóng gói bởi Vite, một công cụ
    build tool thế hệ mới sử dụng ES Modules để tối ưu hóa tốc độ tải
    trang.
  \end{itemize}
\item
  Tầng Server (Backend):

  \begin{itemize}
  \item
    Đóng vai trò là trung tâm xử lý logic (Business Logic Center).
    Server cung cấp các RESTful API để Client giao tiếp.
  \item
    Điểm đặc biệt trong kiến trúc Backend của Dana Travel là sự áp dụng
    triệt để mô hình Middleware Pipeline của Express.js. Mọi yêu cầu
    (Request) đi vào hệ thống đều phải tuần tự đi qua các lớp lọc:
    Logger (Ghi nhật ký) -\textgreater{} CORS (Bảo mật nguồn gốc)
    -\textgreater{} Body Parser (Phân tích dữ liệu) -\textgreater{}
    Router (Định tuyến) -\textgreater{} Controller.
  \end{itemize}
\item
  Tầng Dữ liệu (Persistence Layer):

  \begin{itemize}
  \item
    Sử dụng SQLite làm động cơ lưu trữ (Storage Engine).
  \item
    Sử dụng Prisma ORM làm lớp trừu tượng hóa dữ liệu (Data Abstraction
    Layer). Thay vì Client hay Server giao tiếp trực tiếp với file .db,
    mọi thao tác đều phải thông qua Prisma Client để đảm bảo an toàn
    kiểu dữ liệu (Type Safety).
  \end{itemize}

  \begin{enumerate}
  \def\labelenumi{\arabic{enumi}.}
  \item
    \protect\phantomsection\label{_Toc215924042}{}\textbf{Thiết kế chi
    tiết kiến trúc Backend (Layered Architecture)}
  \end{enumerate}
\end{itemize}

Để tránh tình trạng "Spaghetti Code" (Mã nguồn rối rắm), Backend được tổ
chức nghiêm ngặt theo mô hình 3 lớp (3-Tier Architecture):

\begin{enumerate}
\def\labelenumi{\arabic{enumi}.}
\item
  \protect\phantomsection\label{_Toc215924043}{}\textbf{Lớp Định tuyến
  và Điều khiển (Controller Layer)}
\end{enumerate}

\begin{itemize}
\item
  Đây là lớp giao tiếp (Interface Layer). Nhiệm vụ duy nhất của nó là
  tiếp nhận HTTP Request, giải mã các tham số (req.body, req.query) và
  thực hiện kiểm tra sơ bộ (Basic Validation).
\item
  Ví dụ minh họa: Trong itinerary.controller.js, hàm generateItinerary
  chỉ kiểm tra xem người dùng có gửi lên duration hay không. Nếu có, nó
  chuyển quyền xử lý cho Service. Nó tuyệt đối không chứa logic thuật
  toán xếp lịch.

  \begin{enumerate}
  \def\labelenumi{\arabic{enumi}.}
  \item
    \protect\phantomsection\label{_Toc215924044}{}\textbf{Lớp Dịch vụ
    (Service Layer)}
  \end{enumerate}
\end{itemize}

\begin{itemize}
\item
  Đây là nơi chứa toàn bộ \textbf{Logic Nghiệp vụ (Business Logic)}.
\item
  Các thuật toán phức tạp như generateDayScheduleStrict (xếp lịch 7 pha)
  hay logic gọi AI RAG đều nằm ở đây.
\item
  Lớp Service hoạt động độc lập với giao thức HTTP. Tức là, nếu sau này
  chuyển từ Web sang Mobile App hay CLI (Command Line Interface), logic
  trong Service vẫn tái sử dụng được 100\% mà không cần sửa đổi.

  \begin{enumerate}
  \def\labelenumi{\arabic{enumi}.}
  \item
    \protect\phantomsection\label{_Toc215924045}{}\textbf{Lớp Truy cập
    Dữ liệu (Data Access Layer / Repository)}
  \end{enumerate}
\end{itemize}

\begin{itemize}
\item
  Trong đồ án này, Prisma Client đóng vai trò là lớp Repository.
\item
  Nó ẩn đi sự phức tạp của câu lệnh SQL. Thay vì viết SELECT * FROM
  Location WHERE type = \textquotesingle FOOD\textquotesingle, lập trình
  viên chỉ cần gọi prisma.location.findMany(\{ where: \{ type:
  \textquotesingle FOOD\textquotesingle{} \} \}). Điều này giúp code dễ
  đọc hơn và tránh lỗi SQL Injection.

  \begin{enumerate}
  \def\labelenumi{\arabic{enumi}.}
  \item
    \protect\phantomsection\label{_Toc215924046}{}\textbf{Thiết kế Kiến
    trúc Frontend (Component-Based Architecture)}
  \end{enumerate}
\end{itemize}

Phía Client được thiết kế theo tư duy "Feature-based Folder Structure"
(Cấu trúc thư mục theo tính năng), thay vì chia theo loại file (không
gom tất cả css vào 1 chỗ, js vào 1 chỗ).

\begin{itemize}
\item
  Cấu trúc thư mục:

  \begin{itemize}
  \item
    src/features/itinerary: Chứa tất cả code liên quan đến lịch trình
    (Component hiển thị thẻ, Logic gọi API tạo lịch).
  \item
    src/features/bot: Chứa giao diện chat và logic xử lý tin nhắn.
  \item
    src/features/trip-form: Chứa form nhập liệu.
  \end{itemize}
\item
  Lợi ích: Kiến trúc này giúp dự án dễ dàng mở rộng (Scalable). Khi muốn
  thêm tính năng mới (ví dụ: Booking), chỉ cần tạo thư mục
  src/features/booking mà không làm ảnh hưởng đến các phần khác.

  \begin{enumerate}
  \def\labelenumi{\arabic{enumi}.}
  \item
    \protect\phantomsection\label{_Toc215924047}{}\textbf{Chiến lược Mô
    hình hóa Dữ liệu}
  \end{enumerate}
\end{itemize}

Hệ thống sử dụng chiến lược "Code-First" thông qua Prisma ORM. Thay vì
tạo bảng thủ công bằng lệnh SQL CREATE TABLE, em định nghĩa cấu trúc dữ
liệu bằng ngôn ngữ Schema của Prisma (schema.prisma). Cách tiếp cận này
giúp quản lý phiên bản cơ sở dữ liệu (Database Version Control) dễ dàng
thông qua các file Migration.

\begin{enumerate}
\def\labelenumi{\arabic{enumi}.}
\setcounter{enumi}{1}
\item
  \protect\phantomsection\label{_Toc215924048}{}\textbf{Đặc tả chi tiết
  các Bảng dữ liệu (Table Specifications)}
\end{enumerate}

Dưới đây là phân tích sâu về các quyết định thiết kế cho từng bảng:

\begin{enumerate}
\def\labelenumi{\arabic{enumi}.}
\item
  \protect\phantomsection\label{_Toc215924049}{}\textbf{Bảng Location
  (Địa điểm) Đây là bảng trung tâm (Core Table).}
\end{enumerate}

\begin{itemize}
\item
  id (String - UUID): Sử dụng UUID giúp che giấu tổng số lượng bản ghi
  của hệ thống, tăng tính bảo mật so với ID tự tăng.
\item
  type (Enum LocationType):

  \begin{itemize}
  \item
    Quyết định thiết kế: Sử dụng Enum thay vì String.
  \item
    Lý do: Giới hạn miền giá trị chỉ trong các loại ATTRACTION, FOOD,
    BEACH, ACCOMMODATION, ACTIVITY. Điều này ngăn chặn việc nhập liệu
    sai (ví dụ: nhập "FOD" thay vì "FOOD") làm hỏng thuật toán lọc.
  \end{itemize}
\item
  priceLevel (Enum PriceLevel):

  \begin{itemize}
  \item
    Ý nghĩa: Phân loại mức giá thành BUDGET (Rẻ), MODERATE (Vừa), LUXURY
    (Đắt).
  \item
    Ứng dụng: Thuật toán xếp lịch sử dụng trường này để khớp với ngân
    sách người dùng chọn.
  \end{itemize}
\item
  averagePrice (Int): Lưu giá trị số học để tính tổng chi phí chuyến đi.
\item
  openingTime/closingTime (String): Lưu dưới dạng chuỗi "HH:mm". Hệ
  thống có một module tiện ích time.utils.js để chuyển đổi chuỗi này
  thành số phút trong ngày (Minutes from midnight) phục vụ việc so sánh
  logic (ví dụ: 08:00 = 480 phút).

  \begin{enumerate}
  \def\labelenumi{\arabic{enumi}.}
  \item
    \protect\phantomsection\label{_Toc215924050}{}\textbf{Bảng Knowledge
    (Tri thức AI)}
  \end{enumerate}
\end{itemize}

\begin{itemize}
\item
  Mục đích: Lưu trữ dữ liệu phi cấu trúc (Unstructured Data) để phục vụ
  Chatbot.
\item
  topic (String): Chủ đề tri thức (Ví dụ: "Weather", "Culture").
\item
  content (String): Nội dung văn bản dài (Text Blob).
\item
  Cơ chế hoạt động: Khi người dùng hỏi một câu không liên quan đến địa
  điểm cụ thể (ví dụ: "Đà Nẵng mùa này có mưa không?"), hệ thống sẽ thực
  hiện Full-Text Search trên trường content của bảng này để tìm câu trả
  lời.

  \begin{enumerate}
  \def\labelenumi{\arabic{enumi}.}
  \item
    \protect\phantomsection\label{_Toc215924051}{}\textbf{Bảng Admin
    (Quản trị viên)}
  \end{enumerate}
\end{itemize}

\begin{itemize}
\item
  username (String): Tên đăng nhập (Unique).
\item
  passwordHash (String): Mật khẩu đã mã hóa.
\item
  Bảo mật: Tuyệt đối không lưu mật khẩu thô. Hệ thống sử dụng thuật toán
  băm (Hashing) một chiều. Khi Admin đăng nhập, mật khẩu nhập vào sẽ
  được băm lại và so sánh với chuỗi hash trong DB.

  \begin{enumerate}
  \def\labelenumi{\arabic{enumi}.}
  \item
    \protect\phantomsection\label{_Toc215924052}{}\textbf{Bảng AccessLog
    (Nhật ký truy cập)}
  \end{enumerate}
\end{itemize}

\begin{itemize}
\item
  endpoint (String): Đường dẫn API được gọi.
\item
  method (String): GET/POST.
\item
  timestamp (DateTime): Thời điểm gọi.
\item
  Chiến lược ghi log: Bảng này được thiết kế độc lập, không có khóa
  ngoại (Foreign Key

  \begin{enumerate}
  \def\labelenumi{\arabic{enumi}.}
  \item
    \protect\phantomsection\label{_Toc215924053}{}\textbf{Chiến lược Tối
    ưu hóa Hiệu năng Cơ sở dữ liệu (Database Performance Strategy)}
  \end{enumerate}
\end{itemize}

Vì hệ thống Dana Travel có đặc thù là Read-Heavy (Tác vụ đọc chiếm 90\%
so với tác vụ ghi), đặc biệt là khi thuật toán xếp lịch phải quét qua
bảng Location liên tục, việc tối ưu hóa truy vấn là bắt buộc để đạt chỉ
tiêu phản hồi dưới 2 giây.

\begin{enumerate}
\def\labelenumi{\arabic{enumi}.}
\item
  \protect\phantomsection\label{_Toc215924054}{}\textbf{Chiến lược đánh
  chỉ mục (Indexing Strategy)}
\end{enumerate}

\begin{itemize}
\item
  Mặc dù SQLite là cơ sở dữ liệu dạng file, việc đánh chỉ mục vẫn mang
  lại hiệu quả vượt trội. Dựa trên tần suất truy vấn của thuật toán
  generate Day Schedule Strict, hệ thống thiết kế các chỉ mục (Index)
  trên các trường sau:

  \begin{itemize}
  \item
    Index trên type: Do 100\% các truy vấn lọc địa điểm đều bắt đầu bằng
    điều kiện WHERE type = ... (Ví dụ: Lấy danh sách quán ăn, lấy danh
    sách khách sạn).
  \item
    Index trên priceLevel: Phục vụ cho các bộ lọc ngân sách.
  \end{itemize}
\item
  Việc này giúp độ phức tạp tìm kiếm giảm từ \$O(N)\$ (Quét toàn bộ bảng
  - Full Table Scan) xuống còn \$O(log N)\$ (Tìm kiếm nhị phân trên cây
  B-Tree).

  \begin{enumerate}
  \def\labelenumi{\arabic{enumi}.}
  \item
    \protect\phantomsection\label{_Toc215924055}{}\textbf{Quản lý kết
    nối (Connection Management)}
  \end{enumerate}
\end{itemize}

\begin{itemize}
\item
  Hệ thống sử dụng mẫu thiết kế \textbf{Singleton} cho PrismaClient
  trong file src/utils/prisma.js.
\item
  Cơ chế: Thay vì khởi tạo một kết nối mới đến file dev.db mỗi khi có
  request (gây tốn tài nguyên I/O), hệ thống chỉ khởi tạo một thể hiện
  (Instance) duy nhất và tái sử dụng nó cho toàn bộ vòng đời ứng dụng.
  Điều này ngăn chặn lỗi "Too many open files" khi lượng người dùng tăng
  cao.

  \begin{enumerate}
  \def\labelenumi{\arabic{enumi}.}
  \item
    \protect\phantomsection\label{_Toc215924056}{}\textbf{Tối ưu hóa
    truy vấn dữ liệu (Query Optimization)}
  \end{enumerate}
\end{itemize}

\begin{itemize}
\item
  Selective Projection (Chọn lọc trường): Khi thuật toán sinh lịch chạy,
  nó chỉ cần các trường id, name, type, priceLevel, openingTime. Hệ
  thống cấu hình Prisma chỉ SELECT đúng các cột này, bỏ qua cột
  description (vốn chứa văn bản dài). Việc này giảm thiểu đáng kể dung
  lượng dữ liệu truyền từ ổ cứng lên RAM, giúp thuật toán chạy nhanh
  hơn.
\item
  Batch Processing (Xử lý theo lô): Khi cần lấy thông tin chi tiết cho
  lịch trình 5 ngày (khoảng 15-20 địa điểm), thay vì gọi 20 câu lệnh
  SELECT riêng lẻ, hệ thống gom ID lại và dùng toán tử IN (SELECT * FROM
  Location WHERE id IN (...)) để lấy dữ liệu trong một lần truy xuất duy
  nhất.

  \begin{enumerate}
  \def\labelenumi{\arabic{enumi}.}
  \item
    \protect\phantomsection\label{_Toc215924057}{}\textbf{Định nghĩa bài
    toán và Không gian tìm kiếm}

    \begin{enumerate}
    \def\labelenumii{\arabic{enumii}.}
    \item
      \protect\phantomsection\label{_Toc215924058}{}\textbf{Mô hình toán
      học:}
    \end{enumerate}
  \end{enumerate}
\end{itemize}

Bài toán đặt ra là sắp xếp một tập hợp các địa điểm
\(L\  = \ \backslash\{ l\_ 1,\ l\_ 2,\ ...,\ l\_ n\backslash\}\) vào các
khe thời gian S =
\{\(s\_\{ morning\},\ s\_\{ lunch\},\ s\_\{ afternoon\)\}, ...\} của D
ngày. Mục tiêu là tìm ra một cấu hình C sao cho tối đa hóa sự hài lòng
của người dùng dựa trên sở thích (Interests) và ngân sách (Budget), đồng
thời không vi phạm các ràng buộc cứng.

\begin{enumerate}
\def\labelenumi{\arabic{enumi}.}
\setcounter{enumi}{1}
\item
  \protect\phantomsection\label{_Toc215924059}{}\textbf{Các ràng buộc
  cứng (Hard Constraints):}
\end{enumerate}

Thuật toán buộc phải tuân thủ các quy tắc sau, nếu vi phạm, lịch trình
bị coi là không hợp lệ (Invalid):

\begin{itemize}
\item
  C\_\{time\} (Ràng buộc thời gian): Địa điểm l được chọn vào khe \$s\$
  phải thỏa mãn: \(OpenTime(l)\  \leq Time(s)\  \leq CloseTime(l)\). Ví
  dụ: Không thể xếp Chợ đêm Sơn Trà (mở 18:00) vào buổi sáng.
\item
  C\_\{category\} (Ràng buộc loại hình): Loại hình của địa điểm phải
  khớp với ngữ nghĩa của khe thời gian.
  \(Type(l)\  \in \ AllowedTypes(s).\) Ví dụ: Khe LUNCH bắt buộc phải là
  FOOD.
\item
  C\_\{unique\} (Ràng buộc duy nhất): Một địa điểm không được xuất hiện
  quá 1 lần trong toàn bộ chuyến đi. \(Count(l)\  \leq 1\).

  \begin{enumerate}
  \def\labelenumi{\arabic{enumi}.}
  \item
    \protect\phantomsection\label{_Toc215924060}{}\textbf{Thiết kế chi
    tiết Giải thuật 7 Pha (The 7-Phase Algorithm)}
  \end{enumerate}
\end{itemize}

Thay vì sử dụng các thuật toán quay lui (Backtracking) phức tạp có thể
gây chậm trễ, em thiết kế một quy trình tuyến tính gồm 7 pha xử lý tuần
tự cho mỗi ngày. Mỗi pha đóng vai trò như một bộ lọc (Filter Pipeline)
để chọn ra ứng viên tốt nhất.

Dưới đây là phân tích logic chi tiết từng pha dựa trên mã nguồn:

\begin{itemize}
\item
  Pha 1: Bữa sáng (Breakfast Selection)

  \begin{itemize}
  \item
    Input: Danh sách toàn bộ địa điểm FOOD.
  \item
    Logic lọc:
    location.tags.includes(\textquotesingle breakfast\textquotesingle).
    Hệ thống ưu tiên chọn các món ăn nước (Bún, Phở, Mì Quảng) thay vì
    cơm hay hải sản vào buổi sáng.
  \item
    Cơ chế Random có trọng số: Nếu người dùng chọn ngân sách BUDGET, các
    quán ăn vỉa hè sẽ có xác suất được chọn cao hơn.
  \end{itemize}
\item
  Pha 2: Hoạt động chính buổi sáng (Morning Core Activity)

  \begin{itemize}
  \item
    Chiến lược Heuristic: "Đi xa trước, về gần sau". Thuật toán ưu tiên
    chọn các địa điểm tham quan (ATTRACTION) nằm xa trung tâm hoặc các
    bãi biển (BEACH) vào buổi sáng để tận dụng thời tiết mát mẻ và ánh
    sáng tốt để chụp ảnh.
  \item
    Kiểm tra giờ mở cửa: Hệ thống gọi hàm tiện ích isValidOpenTime(loc,
    \textquotesingle08:30\textquotesingle) để đảm bảo địa điểm đã mở cửa
    đón khách.
  \end{itemize}
\item
  Pha 3: Bữa trưa (Lunch)

  \begin{itemize}
  \item
    Ràng buộc: Phải là FOOD.
  \item
    Kiểm tra trùng lặp: Hệ thống kiểm tra ID quán ăn trưa với ID quán ăn
    sáng đã chọn ở Pha 1. Nếu trùng, thuật toán sẽ lặp lại bước chọn
    (Retry) tối đa 3 lần để tìm quán khác.
  \end{itemize}
\item
  Pha 4: Hoạt động chính buổi chiều (Afternoon Core Activity)

  \begin{itemize}
  \item
    Bộ nhớ trạng thái (State Memory): Hệ thống duy trì một biến
    usedLocationIds (kiểu dữ liệu Set) xuyên suốt quá trình sinh lịch.
    Trước khi chốt địa điểm chiều, thuật toán kiểm tra: if
    (usedLocationIds.has(candidate.id)) continue;. Điều này đảm bảo dù
    đi 5 hay 7 ngày, khách cũng không bao giờ bị gợi ý lặp lại một địa
    điểm.
  \item
    Đa dạng hóa: Nếu buổi sáng đã đi Biển, thuật toán sẽ ưu tiên chọn
    Bảo tàng hoặc Khu vui chơi trong nhà vào buổi chiều để cân bằng trải
    nghiệm.
  \end{itemize}
\item
  Pha 5: Bữa tối (Dinner)

  \begin{itemize}
  \item
    Logic Ngân sách (Budget Logic): Đây là pha nhạy cảm nhất về giá.

    \begin{itemize}
    \item
      Nếu Budget == LUXURY: Bộ lọc chỉ giữ lại các nhà hàng có
      priceLevel == LUXURY.
    \item
      Nếu Budget == BUDGET: Bộ lọc tìm các quán BUDGET hoặc MODERATE.
    \end{itemize}
  \item
    Fail-safe: Trong trường hợp không tìm thấy nhà hàng đúng mức giá (ví
    dụ: khu vực đó toàn nhà hàng đắt đỏ), thuật toán sẽ tự động "hạ
    chuẩn" hoặc "nâng chuẩn" (Relaxation) để đảm bảo người dùng vẫn có
    chỗ ăn tối thay vì để trống lịch trình.
  \end{itemize}
\item
  Pha 6: Hoạt động Tối (Evening/Nightlife)

  \begin{itemize}
  \item
    Bộ lọc đặc biệt: Thuật toán chỉ xét các địa điểm có loại hình
    ACTIVITY (Hoạt động) hoặc COFFEE (Quán cafe).
  \item
    Logic thời gian: Tự động loại bỏ các địa điểm đóng cửa trước 21:00.
    Ưu tiên các địa điểm có tag nightlife (như Chợ đêm, Du thuyền, Cầu
    Rồng).
  \end{itemize}
\item
  Pha 7: Về nghỉ (Return)

  \begin{itemize}
  \item
    Kết thúc chu trình xử lý một ngày, cập nhật lại danh sách
    usedLocationIds và chuyển sang ngày tiếp theo (Day N+1).
  \end{itemize}

  \begin{enumerate}
  \def\labelenumi{\arabic{enumi}.}
  \item
    \protect\phantomsection\label{_Toc215924061}{}\textbf{Cơ chế
    Fallback và Xử lý ngoại lệ (Fallback Mechanism)}
  \end{enumerate}
\end{itemize}

Một điểm yếu của các thuật toán lọc cứng (Strict Filtering) là dễ rơi
vào trạng thái "Không có kết quả" (Empty Result). Để khắc phục, em cài
đặt cơ chế Fallback đa tầng:

\begin{itemize}
\item
  Tầng 1 (Strict Match): Tìm chính xác theo Type + Price + Tag.
\item
  Tầng 2 (Price Relaxation): Nếu Tầng 1 rỗng, bỏ điều kiện Price, chỉ
  tìm theo Type + Tag. (Chấp nhận ăn đắt hơn hoặc rẻ hơn dự kiến một
  chút).
\item
  Tầng 3 (Tag Relaxation): Nếu Tầng 2 rỗng, bỏ tiếp Tag, chỉ tìm theo
  Type. (Ví dụ: Không tìm được quán có tag "Lãng mạn" thì chọn quán ăn
  thường).
\item
  Tầng 4 (Popularity Default): Nếu tất cả đều thất bại, hệ thống chọn
  địa điểm phổ biến nhất trong Database để lấp vào chỗ trống.

  \begin{enumerate}
  \def\labelenumi{\arabic{enumi}.}
  \item
    \protect\phantomsection\label{_Toc215924062}{}\textbf{Kiến trúc
    Module AI (AI Module Architecture)}
  \end{enumerate}
\end{itemize}

Module này được tách biệt trong chatbot.service.js và gemini.adapter.js,
hoạt động theo quy trình 3 bước: Hiểu - Tra cứu - Phản hồi.

\begin{enumerate}
\def\labelenumi{\arabic{enumi}.}
\item
  \protect\phantomsection\label{_Toc215924063}{}\textbf{Tầng Adapter
  (Lớp thích ứng):}
\end{enumerate}

\begin{itemize}
\item
  Em sử dụng mẫu thiết kế Adapter (Adapter Pattern) để đóng gói việc
  giao tiếp với Google Gemini API.
\item
  Lợi ích: Nếu sau này muốn chuyển sang sử dụng OpenAI (ChatGPT) hay
  Claude, em chỉ cần viết lại file openai.adapter.js mà không cần sửa
  đổi logic nghiệp vụ chính. Điều này đảm bảo tính linh hoạt và khả năng
  bảo trì lâu dài cho hệ thống.

  \begin{enumerate}
  \def\labelenumi{\arabic{enumi}.}
  \item
    \protect\phantomsection\label{_Toc215924064}{}\textbf{Cơ chế RAG
    (Retrieval Augmented Generation):}
  \end{enumerate}
\end{itemize}

Đây là kỹ thuật cốt lõi giúp Chatbot thông minh hơn các bot chat thông
thường.

\begin{itemize}
\item
  Bước 1 - Trích xuất từ khóa (Keyword Extraction): Khi người dùng hỏi
  "Giá vé Bà Nà bao nhiêu?", hệ thống không gửi ngay câu hỏi này cho
  Google. Trước hết, nó phân tích và trích xuất từ khóa "Bà Nà".
\item
  Bước 2 - Truy vấn ngữ cảnh (Context Retrieval): Hệ thống thực hiện
  truy vấn vào cơ sở dữ liệu nội bộ (Bảng Location): SELECT * FROM
  Location WHERE name LIKE \textquotesingle\%Bà Nà\%\textquotesingle.
  Kết quả trả về là một bản ghi dữ liệu thô: \{name: "Bà Nà Hills",
  price: 900000, open: "07:00"\}.
\item
  Bước 3 - Tiêm ngữ cảnh (Context Injection): Dữ liệu thô này được
  chuyển đổi thành văn bản tự nhiên và ghép vào câu lệnh (Prompt) gửi
  cho AI.

  \begin{enumerate}
  \def\labelenumi{\arabic{enumi}.}
  \item
    \protect\phantomsection\label{_Toc215924065}{}\textbf{Kỹ thuật
    Prompt Engineering (Kỹ nghệ ra lệnh cho AI)}
  \end{enumerate}
\end{itemize}

Chất lượng câu trả lời của AI phụ thuộc 80\% vào cách chúng ta đặt câu
lệnh (Prompt). Trong file gemini.adapter.js, em đã thiết kế một khuôn
mẫu Prompt (System Prompt Template) chuyên biệt cho du lịch:

"Bạn là DanaAI, một trợ lý du lịch ảo chuyên nghiệp, thân thiện và am
hiểu về Đà Nẵng.

Nhiệm vụ của bạn là trả lời câu hỏi của người dùng dựa trên THÔNG TIN
ĐƯỢC CUNG CẤP dưới đây.

{[}THÔNG TIN NGỮ CẢNH TỪ DATABASE{]}

- Địa điểm: Bà Nà Hills

- Giá vé: 900.000 VNĐ

- Giờ mở cửa: 07:00 - 22:00

{[}YÊU CẦU BẮT BUỘC{]}

1. Chỉ sử dụng thông tin trong phần ngữ cảnh để trả lời.

2. Nếu thông tin không có trong ngữ cảnh, hãy xin lỗi và nói rằng bạn
chưa có dữ liệu đó. Tuyệt đối không tự bịa ra thông tin.

3. Trả lời ngắn gọn, súc tích, giọng điệu vui vẻ.

4. Định dạng câu trả lời bằng Markdown (dùng dấu * để liệt kê ý)."

\begin{itemize}
\item
  Phân tích chiến lược Prompt:

  \begin{itemize}
  \item
    Persona (Nhân cách): Định hình AI là "DanaAI", tạo cảm giác gần gũi
    thương hiệu.
  \item
    Grounding (Neo thông tin): Yêu cầu số 1 và 2 buộc AI phải "neo" câu
    trả lời vào dữ liệu thật, loại bỏ hoàn toàn khả năng chém gió sai sự
    thật.
  \item
    Formatting (Định dạng): Yêu cầu số 4 giúp câu trả lời hiển thị đẹp
    mắt trên giao diện Frontend (có gạch đầu dòng, bôi đậm).
  \end{itemize}

  \begin{enumerate}
  \def\labelenumi{\arabic{enumi}.}
  \item
    \protect\phantomsection\label{_Toc215924066}{}\textbf{Quản lý lịch
    sử hội thoại (Conversation History Management)}
  \end{enumerate}
\end{itemize}

\begin{itemize}
\item
  Để Chatbot có thể hiểu được mạch câu chuyện (Context Window), hệ thống
  lưu trữ tạm thời lịch sử chat của phiên làm việc.

  \begin{itemize}
  \item
    Khi người dùng hỏi câu tiếp theo: "Ở đó có gì chơi?", AI sẽ tự hiểu
    "Ở đó" chính là "Bà Nà Hills" đã nhắc đến ở câu trước.
  \item
    Tối ưu hóa: Do giới hạn token của API Gemini, em chỉ gửi kèm 5-10
    cặp câu hỏi-đáp gần nhất để đảm bảo tốc độ phản hồi nhanh và tiết
    kiệm chi phí API.
  \end{itemize}

  \begin{enumerate}
  \def\labelenumi{\arabic{enumi}.}
  \item
    \protect\phantomsection\label{_Toc215924067}{}\textbf{Chiến lược
    thiết kế trải nghiệm (UX Strategy)}
  \end{enumerate}
\end{itemize}

\begin{itemize}
\item
  Triết lý Mobile-First: Dựa trên phân tích chân dung người dùng
  (Persona) ở Phần 1, đa số khách du lịch sử dụng điện thoại khi đang di
  chuyển. Do đó, hệ thống sử dụng Framework Tailwind CSS với các lớp
  tiện ích Responsive (w-full, md:w-1/2) để đảm bảo giao diện không bị
  vỡ trên màn hình nhỏ.
\item
  Quy tắc "3 Click": Người dùng phải nhận được kết quả (Lịch trình)
  trong tối đa 3 lần nhấp chuột từ Trang chủ:
\end{itemize}

\begin{enumerate}
\def\labelenumi{\arabic{enumi}.}
\item
  Click chọn Số ngày \& Sở thích.
\item
  Click nút "Tạo lịch trình".
\item
  Kết quả hiển thị ngay lập tức.

  \begin{enumerate}
  \def\labelenumii{\arabic{enumii}.}
  \item
    \protect\phantomsection\label{_Toc215924068}{}\textbf{Thiết kế chi
    tiết các màn hình (Screen Design)}

    \begin{enumerate}
    \def\labelenumiii{\arabic{enumiii}.}
    \item
      \protect\phantomsection\label{_Toc215924069}{}\textbf{Màn hình
      Nhập liệu thông minh (Smart Input Form)}
    \end{enumerate}
  \end{enumerate}
\end{enumerate}

\begin{itemize}
\item
  Vấn đề: Các form truyền thống bắt người dùng nhập tay (text input) rất
  dễ sai sót và gây nản lòng.
\item
  Giải pháp:

  \begin{itemize}
  \item
    Duration Slider: Sử dụng thanh trượt (range input) cho việc chọn số
    ngày (1 đến 7). Điều này ngăn chặn việc nhập số âm hoặc ký tự lạ.
  \item
    Interest Chips: Sở thích (Ăn uống, Văn hóa...) được hiển thị dưới
    dạng các "viên thuốc" (Chips/Tags) có thể chọn/bỏ chọn nhanh chóng.
  \item
    Visual Feedback: Khi người dùng thay đổi ngân sách (Ví dụ: từ Thấp
    lên Cao), giao diện sẽ hiển thị gợi ý tức thì ("Gợi ý: Phù hợp cho
    nghỉ dưỡng sang trọng") để người dùng hiểu rõ lựa chọn của mình.

    \begin{enumerate}
    \def\labelenumi{\arabic{enumi}.}
    \item
      \protect\phantomsection\label{_Toc215924070}{}\textbf{Màn hình Kết
      quả Lịch trình (Itinerary Timeline)}
    \end{enumerate}
  \end{itemize}
\end{itemize}

\begin{itemize}
\item
  Cấu trúc hiển thị: Sử dụng mô hình Vertical Timeline (Dòng thời gian
  dọc).
\item
  Thẻ thông tin (Itinerary Card): Mỗi hoạt động là một thẻ độc lập chứa:

  \begin{itemize}
  \item
    Ảnh đại diện: (Lấy từ Database) giúp kích thích thị giác.
  \item
    Giờ giấc: (08:00 - 10:00) được làm nổi bật.
  \item
    Huy hiệu (Badge): Hiển thị loại hình (FOOD màu cam, BEACH màu xanh
    dương) giúp người dùng lướt nhanh (scan) nội dung dễ dàng.
  \end{itemize}
\item
  Tính năng tương tác: Người dùng có thể bấm vào tên địa điểm để mở ra
  cửa sổ chi tiết (Modal) xem địa chỉ bản đồ hoặc giá vé cụ thể.

  \begin{enumerate}
  \def\labelenumi{\arabic{enumi}.}
  \item
    \protect\phantomsection\label{_Toc215924071}{}\textbf{Giao diện
    Chatbot (Conversational UI)}
  \end{enumerate}
\end{itemize}

\begin{itemize}
\item
  Vị trí: Nút nổi (Floating Action Button) ở góc dưới bên phải màn hình,
  luôn hiển thị đè lên mọi nội dung khác.
\item
  Hiệu ứng:

  \begin{itemize}
  \item
    Typing Indicator: Khi chờ AI xử lý (thường mất 2-3s), hiển thị hiệu
    ứng "ba chấm nhấp nháy" để người dùng biết hệ thống không bị treo.
  \item
    Markdown Rendering: Câu trả lời của AI được định dạng đẹp mắt (in
    đậm, gạch đầu dòng) thay vì một khối văn bản dày đặc, giúp thông tin
    dễ đọc hơn.
  \end{itemize}

  \begin{enumerate}
  \def\labelenumi{\arabic{enumi}.}
  \item
    \protect\phantomsection\label{_Toc215924072}{}\textbf{Cơ chế Bảo mật
    đa lớp (Layered Security)}

    \begin{enumerate}
    \def\labelenumii{\arabic{enumii}.}
    \item
      \protect\phantomsection\label{_Toc215924073}{}\textbf{Bảo mật ở
      mức Ứng dụng (Application Level)}
    \end{enumerate}
  \end{enumerate}
\end{itemize}

\begin{itemize}
\item
  Input Validation (Kiểm tra đầu vào): Mọi dữ liệu gửi lên từ Client đều
  bị coi là "không đáng tin". Tại itinerary.controller.js, hệ thống kiểm
  tra kiểu dữ liệu (Type Check) và phạm vi giá trị (Range Check). Ví dụ:
  Nếu budget gửi lên là chuỗi script độc hại
  \textless script\textgreater alert(1)\textless/script\textgreater, hệ
  thống sẽ từ chối xử lý ngay lập tức để chống tấn công XSS.
\item
  Rate Limiting: Để chống spam API Chatbot (gây tốn chi phí Gemini), hệ
  thống giới hạn mỗi IP chỉ được gửi tối đa 20 tin nhắn/phút.

  \begin{enumerate}
  \def\labelenumi{\arabic{enumi}.}
  \item
    \protect\phantomsection\label{_Toc215924074}{}\textbf{Bảo mật ở mức
    Quản trị (Admin Security)}
  \end{enumerate}
\end{itemize}

\begin{itemize}
\item
  Password Hashing: Sử dụng thuật toán Bcrypt với salt rounds = 10. Ngay
  cả khi hacker dump được database, họ chỉ thấy chuỗi ký tự vô nghĩa
  \$2b\$10\$EpIy..., không thể dịch ngược ra mật khẩu gốc.
\item
  Authentication Middleware: Middleware adminAuth chặn tất cả các
  request vào /api/admin. Nó kiểm tra token trong Header Authorization.
  Nếu token không hợp lệ hoặc hết hạn, trả về lỗi 401 Unauthorized ngay
  lập tức, không tốn tài nguyên xử lý logic.

  \begin{enumerate}
  \def\labelenumi{\arabic{enumi}.}
  \item
    \protect\phantomsection\label{_Toc215924075}{}\textbf{Chiến lược
    Quản lý Lỗi (Error Handling Strategy)}
  \end{enumerate}
\end{itemize}

Một hệ thống tốt không phải là hệ thống không bao giờ lỗi, mà là hệ
thống biết cách xử lý lỗi một cách duyên dáng (Graceful Degradation).

\begin{enumerate}
\def\labelenumi{\arabic{enumi}.}
\item
  \protect\phantomsection\label{_Toc215924076}{}\textbf{Centralized
  Error Handling (Xử lý lỗi tập trung)}
\end{enumerate}

\begin{itemize}
\item
  Thay vì dùng try-catch lắt nhắt ở khắp nơi, hệ thống sử dụng một
  Middleware xử lý lỗi toàn cục error.handler.middleware.js.
\item
  \emph{Cơ chế:} Khi có lỗi xảy ra ở bất kỳ đâu (DB chết, Logic sai),
  lỗi sẽ được next(error) chuyển về middleware này.
\item
  \emph{Phản hồi:} Middleware sẽ ghi log chi tiết ra file (cho Admin
  xem) và trả về thông báo lỗi chung chung "Đã có lỗi xảy ra, vui lòng
  thử lại" (cho User xem). Điều này giúp che giấu thông tin kỹ thuật
  nhạy cảm (như stack trace) khỏi người dùng cuối.

  \begin{enumerate}
  \def\labelenumi{\arabic{enumi}.}
  \item
    \protect\phantomsection\label{_Toc215924077}{}\textbf{Fallback Logic
    (Logic dự phòng)}
  \end{enumerate}
\end{itemize}

\begin{itemize}
\item
  \emph{Kịch bản:} Khi thuật toán sinh lịch không tìm thấy nhà hàng phù
  hợp (do lọc quá kỹ).
\item
  \emph{Xử lý:} Thay vì trả về màn hình trắng hoặc lỗi 500, hệ thống tự
  động kích hoạt chế độ "Tìm kiếm mở rộng" (như đã phân tích ở Phần 7),
  đảm bảo người dùng luôn nhận được một kết quả khả dĩ nhất.
\end{itemize}

\textbf{\hfill\break
}

\protect\phantomsection\label{citations}{}\textbf{CHƯƠNG 3:} \textbf{KẾT
QUẢ VÀ ỨNG DỤNG}

\begin{enumerate}
\def\labelenumi{\arabic{enumi}.}
\item
  \protect\phantomsection\label{_Toc215924079}{}\textbf{Môi trường và
  công cụ phát triển}
\end{enumerate}

Để đảm bảo tính nhất quán trong quá trình phát triển và vận hành, hệ
thống được xây dựng dựa trên các cấu hình phần cứng và phần mềm cụ thể
như sau:

\begin{enumerate}
\def\labelenumi{\arabic{enumi}.}
\item
  \protect\phantomsection\label{_Toc215924080}{}\textbf{Cấu hình phần
  cứng (Development Environment)}
\end{enumerate}

Hệ thống được phát triển và thử nghiệm trên máy trạm cục bộ (Localhost)
với thông số:

\begin{itemize}
\item
  Vi xử lý (CPU): AMD Ryzen 5 5600H with Radeon Graphics (3.30 GHz)
\item
  Bộ nhớ trong (RAM): 16GB DDR4 (Đảm bảo khả năng chạy đồng thời cả
  Backend Node.js, Frontend Vite server.
\item
  Ổ cứng: SSD NVMe 512GB (Tăng tốc độ đọc ghi dữ liệu Database SQLite).
\item
  Hệ điều hành: Windows 11 Home 64-bit

  \begin{enumerate}
  \def\labelenumi{\arabic{enumi}.}
  \item
    \protect\phantomsection\label{_Toc215924081}{}\textbf{Nền tảng và
    Thư viện Phần mềm Danh sách các công nghệ cốt lõi được sử dụng trong
    mã nguồn:}
  \end{enumerate}
\end{itemize}

\begin{itemize}
\item
  Hệ thống sử dụng bộ công nghệ (Tech Stack) dựa trên hệ sinh thái
  JavaScript:

  \begin{itemize}
  \item
    Phía Máy chủ (Backend):
  \item
    Node.js Runtime: Phiên bản v18.x (LTS). Đây là phiên bản hỗ trợ dài
    hạn, đảm bảo tính ổn định cho các module xử lý bất đồng bộ.
  \item
    Express.js Framework (v4.18): Dùng để xây dựng RESTful API.
  \item
    Prisma ORM (v5.x): Công cụ ánh xạ cơ sở dữ liệu.
  \item
    Google Generative AI SDK: Thư viện chính chủ của Google để kết nối
    với mô hình Gemini Pro.
  \end{itemize}
\end{itemize}

\begin{itemize}
\item
  Phía Giao diện (Frontend):

  \begin{itemize}
  \item
    React (v18.2): Thư viện UI cốt lõi.
  \item
    Vite (v5.x): Công cụ đóng gói (Bundler) thế hệ mới, thay thế cho
    Webpack để tăng tốc độ khởi động server phát triển.
  \item
    Tailwind CSS (v3.4): Framework CSS ưu tiên tiện ích.
  \item
    Lucide React: Bộ icon mã nguồn mở, nhẹ và đồng bộ..
  \end{itemize}

  \begin{enumerate}
  \def\labelenumi{\arabic{enumi}.}
  \item
    \protect\phantomsection\label{_Toc215924082}{}\textbf{Quy trình cài
    đặt và triển khai (deployment guide)}
  \end{enumerate}
\end{itemize}

Để chuyển đổi từ mã nguồn sang ứng dụng chạy thực tế, quy trình cài đặt
được thực hiện nghiêm ngặt theo các bước sau:

Bước 1: Thiết lập cấu hình môi trường (Environment Variables)

Hệ thống yêu cầu các biến môi trường nhạy cảm để hoạt động. Tạo tệp .env
tại thư mục gốc của Backend và cấu hình các thông số:

\# Cấu hình Server

PORT=3000

NODE\_ENV=development

\# Kết nối Cơ sở dữ liệu SQLite

DATABASE\_URL="file:./dev.db"

\# Khóa API cho Google Gemini (Bắt buộc cho Chatbot)

GEMINI\_API\_KEY="AIzaSyA\_xxxxxxxxxxxxxxxxxxxxxx"

\# Cấu hình bảo mật JWT (Cho Admin)

JWT\_SECRET="danatravel\_secret\_key\_2024"

Bước 2: Khởi tạo Cơ sở dữ liệu (Database Migration \& Seeding)

Sử dụng Prisma để đồng bộ cấu trúc bảng từ file schema.prisma vào file
dev.db và nạp dữ liệu mẫu:

\begin{itemize}
\item
  Lệnh tạo bảng: npx prisma migrate dev -\/-name init\_schema
\item
  Lệnh nạp dữ liệu mẫu (Seed): npx prisma db seed

  \begin{itemize}
  \item
    Kết quả: Hệ thống tự động chèn 50 địa điểm du lịch, 5 tài khoản
    admin và 20 câu hỏi tri thức mẫu vào database.
  \end{itemize}
\end{itemize}

Bước 3: Khởi chạy các dịch vụ (Start Services)

Hệ thống hoạt động trên hai tiến trình song song:

\begin{enumerate}
\def\labelenumi{\arabic{enumi}.}
\item
  Backend Service: Tại thư mục /Backend, chạy lệnh npm run dev. Server
  sẽ lắng nghe tại cổng http://localhost:3000 và kết nối thành công đến
  SQLite.
\item
  Frontend Service: Tại thư mục /Frontend, chạy lệnh npm run dev. Vite
  sẽ khởi động Hot-reload Server tại cổng http://localhost:5173.

  \begin{enumerate}
  \def\labelenumii{\arabic{enumii}.}
  \item
    \protect\phantomsection\label{_Toc215924083}{}\textbf{Kết quả hiện
    thực hóa các chức năng (detailed implementation)}

    \begin{enumerate}
    \def\labelenumiii{\arabic{enumiii}.}
    \item
      \protect\phantomsection\label{_Toc215924084}{}\textbf{Phân hệ
      Khách du lịch - Trang chủ (Home Page)}
    \end{enumerate}
  \end{enumerate}
\end{enumerate}

Mô tả giao diện: Giao diện trang chủ được thiết kế theo phong cách tối
giản (Minimalism) với tông màu chủ đạo là Xanh biển (Blue-500) và Trắng,
tạo cảm giác thân thiện của du lịch biển Đà Nẵng.

Thanh điều hướng (Sticky Navbar): Luôn hiển thị ở trên cùng khi cuộn
trang, giúp truy cập nhanh vào các mục "Lập kế hoạch", "Chatbot", "Tin
tức".

Xử lý kỹ thuật: Khi người dùng truy cập, useEffect hook trong React sẽ
gọi API GET /api/locations?featured=true. Backend Controller sẽ truy vấn
bảng Location, lọc lấy 4 bản ghi và trả về JSON. Frontend nhận dữ liệu
và sử dụng hàm map() để render ra danh sách các Component LocationCard.

\begin{quote}
\includegraphics[width=5.65284in,height=2.57909in,alt={A screenshot of a computer AI-generated content may be incorrect.}]{media/image20.png}
\end{quote}

Hình 3.1: Giao diện Trang chủ

\begin{enumerate}
\def\labelenumi{\arabic{enumi}.}
\setcounter{enumi}{1}
\item
  \protect\phantomsection\label{_Toc215924085}{}\textbf{Phân hệ Khách du
  lịch - Lập lịch trình thông minh (Smart Itinerary)}
\end{enumerate}

Đây là chức năng cốt lõi, thể hiện tính "thông minh" của hệ thống.

\begin{itemize}
\item
  Bước A: Nhập liệu (Input)

  \begin{itemize}
  \item
    Người dùng tương tác với một Form khảo sát đa bước (Multi-step
    Form):
  \end{itemize}

  \begin{enumerate}
  \def\labelenumi{\arabic{enumi}.}
  \item
    Ngân sách: Nhập số tiền (VD: 3.000.000 VNĐ). Hệ thống có validation
    chặn số âm.
  \item
    Thời gian: Chọn số ngày đi (VD: 3 ngày).
  \item
    Sở thích (Tags): Chọn các từ khóa như Biển, Văn hóa, Check-in.
  \end{enumerate}
\end{itemize}

\includegraphics[width=2.81269in,height=1.33376in,alt={A screenshot of a computer AI-generated content may be incorrect.}]{media/image21.png}\includegraphics[width=2.54125in,height=1.26218in,alt={A screenshot of a computer AI-generated content may be incorrect.}]{media/image22.png}\includegraphics[width=3.02222in,height=1.48138in,alt={A screenshot of a computer AI-generated content may be incorrect.}]{media/image23.png}\includegraphics[width=2.6558in,height=1.56101in,alt={A screenshot of a computer AI-generated content may be incorrect.}]{media/image24.png}

\includegraphics[width=2.43634in,height=1.77784in,alt={A screenshot of a website AI-generated content may be incorrect.}]{media/image25.png}

Hình 3.2: Giao diện Form nhập thông tin tạo lịch trình

\begin{itemize}
\item
  Bước B: Xử lý Thuật toán (Processing)

  \begin{itemize}
  \item
    Khi nhấn nút "Tạo lịch trình", Frontend gửi POST Request kèm body
    JSON \{budget: 3000000, days: 3, tags: {[}"beach", "culture"{]} \}.
  \end{itemize}
\end{itemize}

\begin{quote}
Tại Backend, thuật toán Greedy thực hiện:
\end{quote}

\begin{enumerate}
\def\labelenumi{\arabic{enumi}.}
\item
  Lọc danh sách địa điểm theo tags.
\item
  Phân bổ ngân sách ăn uống (30\% tổng quỹ).
\item
  Duyệt vòng lặp để chọn điểm tham quan gần nhất (Nearest Neighbor) cho
  từng buổi Sáng/Chiều/Tối.
\item
  Tính toán chi phí di chuyển dự kiến dựa trên bảng giá Transport.
\end{enumerate}

\begin{itemize}
\item
  Bước C: Hiển thị kết quả (Output)

  \begin{itemize}
  \item
    Màn hình kết quả hiển thị một Timeline dọc chi tiết:

    \begin{itemize}
    \item
      Phân chia rõ ràng: Ngày 1 \textgreater{} Buổi Sáng \textgreater{}
      Địa điểm A \textgreater{} Buổi Trưa \textgreater{} Ăn tại quán B.
    \item
      Dashboard tài chính: Hiển thị biểu đồ thanh (Progress Bar) so sánh
      Chi phí dự kiến vs Ngân sách ban đầu. Nếu vượt ngân sách, thanh
      này chuyển màu đỏ cảnh báo.
    \item
      Bản đồ: Tích hợp Google Maps Iframe để hiển thị vị trí của địa
      điểm đang chọn.
    \end{itemize}
  \end{itemize}
\end{itemize}

\begin{quote}
\includegraphics[width=4.27588in,height=2.58462in,alt={A screenshot of a computer AI-generated content may be incorrect.}]{media/image26.png}
\end{quote}

Hình 3.3: Màn hình Kết quả lịch trình chi tiết từng ngày

\begin{enumerate}
\def\labelenumi{\arabic{enumi}.}
\item
  \protect\phantomsection\label{_Toc215924086}{}\textbf{Phân hệ Khách du
  lịch - Trợ lý ảo AI (RAG Chatbot)}
\end{enumerate}

\begin{itemize}
\item
  Mô tả giao diện:
\end{itemize}

\begin{quote}
Chatbot được thiết kế dưới dạng Widget nổi ở góc phải màn hình. Khi mở
ra, giao diện giống các ứng dụng nhắn tin hiện đại (Zalo/Messenger) với
bong bóng chat (Chat bubbles).
\end{quote}

\begin{itemize}
\item
  Luồng xử lý RAG (Retrieval-Augmented Generation):

  \begin{enumerate}
  \def\labelenumi{\arabic{enumi}.}
  \item
    User hỏi: "Vé Bà Nà Hills bao nhiêu tiền?".
  \item
    Backend nhận câu hỏi, thực hiện tìm kiếm Full-text Search trong bảng
    Location và Knowledge.
  \item
    Kết quả tìm thấy: Bản ghi \{ name: "Bà Nà Hills", ticket: 900000 \}.
  \item
    Hệ thống tạo Prompt: "Dựa vào dữ liệu: Bà Nà giá 900k. Hãy trả lời
    câu hỏi: Vé Bà Nà bao nhiêu?".
  \item
    Google Gemini API trả về câu trả lời tự nhiên: "Chào bạn, hiện tại
    vé Bà Nà Hills là 900.000đ nhé.".
  \end{enumerate}
\end{itemize}

Ưu điểm: Nếu hỏi về một địa điểm không có ở Đà Nẵng (VD: Tháp Eiffel),
hệ thống sẽ không tìm thấy dữ liệu trong DB và AI sẽ trả lời "Xin lỗi,
tôi không có thông tin về địa điểm này tại Đà Nẵng", tránh việc AI bịa
đặt thông tin.

\begin{quote}
\includegraphics[width=6.33564in,height=4.74045in,alt={A screenshot of a chat AI-generated content may be incorrect.}]{media/image27.png}
\end{quote}

Hình 3.4: Giao diện hội thoại với Chatbot và minh họa câu trả lời chính
xác

\begin{enumerate}
\def\labelenumi{\arabic{enumi}.}
\item
  \protect\phantomsection\label{_Toc215924087}{}\textbf{Phân hệ Quản trị
  viên (Admin Dashboard)}
\end{enumerate}

\begin{itemize}
\item
  Mô tả giao diện:
\end{itemize}

\begin{itemize}
\item
  Giao diện quản trị sử dụng Layout riêng biệt với Sidebar bên trái chứa
  các menu: Tổng quan, Địa điểm, Tri thức AI, Tài khoản.
\end{itemize}

\begin{itemize}
\item
  Chức năng Thống kê (Analytics):
\end{itemize}

\begin{itemize}
\item
  Sử dụng thư viện Chart.js để vẽ biểu đồ:

  \begin{itemize}
  \item
    Biểu đồ tròn (Pie Chart): Thể hiện tỉ lệ sở thích của người dùng
    (VD: 40\% thích Biển, 30\% thích Núi).
  \item
    Biểu đồ cột (Bar Chart): Thống kê số lượng lịch trình được tạo trong
    7 ngày gần nhất.
  \end{itemize}
\end{itemize}

\begin{itemize}
\item
  Chức năng Quản lý Địa điểm (CRUD):
\end{itemize}

\begin{quote}
Bảng dữ liệu (Data Table) cho phép Admin tìm kiếm, phân trang.
\end{quote}

\begin{itemize}
\item
  Thêm mới: Form cho phép nhập Tên, Giá vé, Tọa độ, Upload link ảnh,
  Nhập Menu (dạng JSON).
\item
  Sửa/Xóa: Cập nhật thông tin địa điểm theo thời gian thực.
\end{itemize}

\begin{quote}
\includegraphics[width=3.98256in,height=3.75771in,alt={A screenshot of a computer AI-generated content may be incorrect.}]{media/image28.png}
\end{quote}

Hình 3.5: Giao diện Dashboard thống kê và Quản lý danh sách địa điểm

\begin{enumerate}
\def\labelenumi{\arabic{enumi}.}
\item
  \protect\phantomsection\label{_Toc215924088}{}\textbf{KỊCH BẢN KIỂM
  THỬ VÀ KẾT QUẢ (SYSTEM TESTING)}
\end{enumerate}

\textbf{Bảng 3.1: Kiểm thử chức năng Tạo lịch trình (Generate
Itinerary)}

{\def\LTcaptype{none} % do not increment counter
\begin{longtable}[]{@{}
  >{\raggedright\arraybackslash}p{(\linewidth - 10\tabcolsep) * \real{0.1065}}
  >{\raggedright\arraybackslash}p{(\linewidth - 10\tabcolsep) * \real{0.1852}}
  >{\raggedright\arraybackslash}p{(\linewidth - 10\tabcolsep) * \real{0.1890}}
  >{\raggedright\arraybackslash}p{(\linewidth - 10\tabcolsep) * \real{0.2047}}
  >{\raggedright\arraybackslash}p{(\linewidth - 10\tabcolsep) * \real{0.2045}}
  >{\raggedright\arraybackslash}p{(\linewidth - 10\tabcolsep) * \real{0.1101}}@{}}
\toprule\noalign{}
\begin{minipage}[b]{\linewidth}\raggedright
\textbf{Mã Test}
\end{minipage} & \begin{minipage}[b]{\linewidth}\raggedright
\textbf{Mô tả kịch bản}
\end{minipage} & \begin{minipage}[b]{\linewidth}\raggedright
\textbf{Dữ liệu đầu vào}
\end{minipage} & \begin{minipage}[b]{\linewidth}\raggedright
\textbf{Kết quả mong đợi}
\end{minipage} & \begin{minipage}[b]{\linewidth}\raggedright
\textbf{Kết quả thực tế}
\end{minipage} & \begin{minipage}[b]{\linewidth}\raggedright
\textbf{Kết luận}
\end{minipage} \\
\midrule\noalign{}
\endhead
\bottomrule\noalign{}
\endlastfoot
IT\_01 & Tạo lịch trình thành công với ngân sách trung bình & Budget:
5.000.000, Days: 3, People: 2, Tags: Biển & Hệ thống sinh lịch trình đủ
3 ngày, ưu tiên các điểm ven biển. Tổng chi phí \textless{} 5tr. & Lịch
trình hợp lý, tổng chi phí 4.2tr & Đạt \\
IT\_02 & Xử lý khi ngân sách quá thấp & Budget: 200.000, Days: 2,
People: 2 & Days: 2, People: 2Hệ thống cảnh báo không đủ ngân sách hoặc
chỉ gợi ý các điểm miễn phí (cầu Rồng, biển). & Thông báo: "Ngân sách
quá thấp để thực hiện chuyến đi 2 ngày". & Đạt \\
IT\_03 & Kiểm tra ràng buộc thời gian & Chọn đi "Chợ đêm Helio" (mở cửa
17h) & Thuật toán phải xếp Chợ đêm vào buổi Tối, không được xếp vào buổi
Sáng. & Chợ đêm Helio xuất hiện ở Slot Tối Ngày 1. & Đạt \\
\end{longtable}
}

\textbf{Bảng 3.2: Kiểm thử chức năng Chatbot RAG}

{\def\LTcaptype{none} % do not increment counter
\begin{longtable}[]{@{}
  >{\raggedright\arraybackslash}p{(\linewidth - 10\tabcolsep) * \real{0.1075}}
  >{\raggedright\arraybackslash}p{(\linewidth - 10\tabcolsep) * \real{0.1846}}
  >{\raggedright\arraybackslash}p{(\linewidth - 10\tabcolsep) * \real{0.1874}}
  >{\raggedright\arraybackslash}p{(\linewidth - 10\tabcolsep) * \real{0.2056}}
  >{\raggedright\arraybackslash}p{(\linewidth - 10\tabcolsep) * \real{0.1961}}
  >{\raggedright\arraybackslash}p{(\linewidth - 10\tabcolsep) * \real{0.1188}}@{}}
\toprule\noalign{}
\begin{minipage}[b]{\linewidth}\raggedright
\textbf{Mã Test}
\end{minipage} & \begin{minipage}[b]{\linewidth}\raggedright
\textbf{Mô tả kịch bản}
\end{minipage} & \begin{minipage}[b]{\linewidth}\raggedright
\textbf{Câu hỏi đầu vào}
\end{minipage} & \begin{minipage}[b]{\linewidth}\raggedright
\textbf{Kết quả mong đợi}
\end{minipage} & \begin{minipage}[b]{\linewidth}\raggedright
\textbf{Kết quả thực tế}
\end{minipage} & \begin{minipage}[b]{\linewidth}\raggedright
\textbf{Kết luận}
\end{minipage} \\
\midrule\noalign{}
\endhead
\bottomrule\noalign{}
\endlastfoot
CB\_01 & Hỏi thông tin giá vé chính xác & "Vé Núi Thần Tài giá bao
nhiêu?" & Bot trả lời đúng giá trong DB (450k) thay vì giá cũ trên mạng.
& "Hiện tại vé Núi Thần Tài là 450.000 VNĐ". & Đạt \\
CB\_02 & Hỏi thông tin ngoài phạm vi & "Thời tiết Hà Nội hôm nay?" & Bot
từ chối trả lời hoặc lái về chủ đề Đà Nẵng. & "Xin lỗi, tôi chỉ hỗ trợ
thông tin du lịch Đà Nẵng". & Đạt \\
CB\_03 & Hỏi về địa chỉ quán ăn & "Địa chỉ Bánh xèo Bà Dưỡng?" & Bot
cung cấp chính xác địa chỉ trong bảng Location. & "Bánh xèo Bà Dưỡng ở
K280/23 Hoàng Diệu...". & Đạt \\
\end{longtable}
}

\begin{enumerate}
\def\labelenumi{\arabic{enumi}.}
\setcounter{enumi}{1}
\item
  \protect\phantomsection\label{_Toc215924089}{}\textbf{Đánh giá hiệu
  năng (performance evaluation)}
\end{enumerate}

Nhóm sử dụng công cụ Google Lighthouse và Postman để đo lường hiệu năng
hệ thống.

\begin{enumerate}
\def\labelenumi{\arabic{enumi}.}
\item
  \protect\phantomsection\label{_Toc215924090}{}\textbf{Hiệu năng phía
  Máy chủ (Backend Latency)}
\end{enumerate}

Thời gian phản hồi trung bình (Average Response Time) của các API chính:

\begin{itemize}
\item
  GET /api/locations: 45ms (Rất nhanh do truy vấn index SQLite).
\item
  POST /api/itinerary/generate: 180ms - 300ms (Chấp nhận được cho thuật
  toán tính toán phức tạp).
\item
  POST /api/chat: 2.5s - 3.5s (Phụ thuộc vào tốc độ phản hồi của Google
  Gemini API).

  \begin{enumerate}
  \def\labelenumi{\arabic{enumi}.}
  \item
    \protect\phantomsection\label{_Toc215924091}{}\textbf{Hiệu năng phía
    Máy khách (Frontend Metrics)}
  \end{enumerate}
\end{itemize}

Điểm số Lighthouse trên trình duyệt Chrome:

\begin{itemize}
\item
  Performance: 92/100.
\item
  Accessibility: 95/100.
\item
  SEO: 100/100.
\item
  Thời gian tải nội dung đầu tiên (FCP): 0.8s.
\end{itemize}

=\textgreater{} Nhận xét: Việc sử dụng kiến trúc SPA và công cụ build
Vite giúp ứng dụng có tốc độ tải trang gần như tức thời, mang lại trải
nghiệm mượt mà cho người dùng.

\newpage
\section{KẾT LUẬN}\label{kux1ebft-luux1eadn}

\textbf{1. Các kết quả đạt được}

Về mặt nghiên cứu lý thuyết:

\begin{itemize}
\item
  Nắm vững kiến trúc phát triển ứng dụng Web hiện đại theo mô hình
  Client-Server tách biệt, sử dụng ReactJS cho phía người dùng và
  Node.js cho phía máy chủ.
\item
  Hiểu và vận dụng thành công kỹ thuật RAG (Retrieval-Augmented
  Generation) để tích hợp các Mô hình ngôn ngữ lớn (LLM) vào bài toán
  thực tế, giúp kiểm soát tính chính xác của dữ liệu đầu ra từ AI.
\item
  Nghiên cứu sâu về bài toán thỏa mãn ràng buộc (CSP) và áp dụng giải
  thuật Tham lam (Greedy) để giải quyết vấn đề lập lịch trình du lịch.
\end{itemize}

Về mặt sản phẩm ứng dụng: Hệ thống DanaTravel đã được cài đặt và vận
hành ổn định với các chức năng chính:

\begin{itemize}
\item
  Chức năng Lập lịch trình thông minh: Hệ thống đã tự động hóa việc sinh
  ra các lộ trình du lịch chi tiết từng buổi (Sáng/Trưa/Chiều/Tối) dựa
  trên ngân sách và sở thích của người dùng. Thuật toán
  generate-day-schedule-strict hoạt động hiệu quả, đảm bảo tổng chi phí
  không vượt quá ngân sách dự kiến và tối ưu hóa khoảng cách di chuyển
  giữa các điểm.
\item
  Chức năng Chatbot AI tư vấn: Xây dựng thành công trợ lý ảo sử dụng
  Google Gemini, có khả năng trả lời chính xác các câu hỏi về giá vé,
  địa chỉ, giờ mở cửa nhờ vào cơ sở dữ liệu tri thức nội bộ, khắc phục
  được hiện tượng "ảo giác" thường gặp ở AI.
\item
  Hệ thống quản trị (Admin Dashboard): Cung cấp công cụ toàn diện cho
  quản trị viên để quản lý danh sách địa điểm (Location), cập nhật tri
  thức cho AI (Knowledge Base) và theo dõi thống kê xu hướng tìm kiếm
  của người dùng.
\end{itemize}

\textbf{2. Hạn chế và hướng phát triển}

Bên cạnh những kết quả đạt được, đồ án vẫn còn một số hạn chế do giới
hạn về thời gian và nguồn lực:

\begin{itemize}
\item
  Dữ liệu bản đồ hiện tại dựa trên khoảng cách đường chim bay, chưa tính
  toán được tình trạng giao thông thực tế.
\item
  Chưa tích hợp cổng thanh toán trực tuyến để hỗ trợ người dùng đặt vé
  ngay trên ứng dụng.
\end{itemize}

Hướng phát triển trong tương lai:

\begin{itemize}
\item
  Tích hợp Google Maps API để tính toán thời gian di chuyển chính xác
  hơn.
\item
  Phát triển phiên bản ứng dụng di động (Mobile App) để tăng tính tiện
  dụng cho du khách khi đang di chuyển.
\item
  Mở rộng tính năng gợi ý dựa trên cộng đồng (User Reviews) để làm phong
  phú thêm dữ liệu hệ thống.
\end{itemize}

Đồ án này là bước khởi đầu quan trọng, tạo tiền đề vững chắc để em tiếp
tục nghiên cứu và phát triển các ứng dụng du lịch thông minh hơn trong
tương lai.

\newpage
\section{TÀI LIỆU THAM KHẢO}\label{tuxe0i-liux1ec7u-tham-khux1ea3o}

\begin{enumerate}
\def\labelenumi{\arabic{enumi}.}
\item
  Chart.js Contributors. (n.d.). Chart.js: Simple yet flexible
  JavaScript charting for designers \& developers. Retrieved from
  \url{https://www.chartjs.org/docs/latest/}
\item
  Express Contributors. (n.d.). Express: Fast, unopinionated, minimalist
  web framework for Node.js. Retrieved from \url{https://expressjs.com/}
\item
  Framer. (n.d.). Framer Motion documentation. Retrieved from
  \url{https://www.framer.com/motion/}
\item
  Google. (n.d.). Gemini API overview. Google AI for Developers.
  Retrieved from \url{https://ai.google.dev/docs}
\item
  Meta. (n.d.). React: The library for web and native user interfaces.
  Retrieved from \url{https://react.dev/reference/react}
\item
  OpenJS Foundation. (n.d.). Node.js documentation. Retrieved from
  \url{https://nodejs.org/en/docs/}
\item
  Prisma Data, Inc. (n.d.). Prisma ORM documentation. Retrieved from
  \url{https://www.prisma.io/docs/}
\item
  SQLite Consortium. (n.d.). SQLite documentation. Retrieved from
  \url{https://www.sqlite.org/docs.html}
\item
  Tailwind Labs. (n.d.). Tailwind CSS documentation. Retrieved from
  \url{https://tailwindcss.com/docs/}
\item
  You, E. (n.d.). Vite: Next generation frontend tooling. Retrieved from
  \url{https://vitejs.dev/guide/}
\end{enumerate}

\newpage
\section{PHỤ LỤC}\label{phux1ee5-lux1ee5c}

\textbf{PHỤ LỤC 1. HƯỚNG DẪN CÀI ĐẶT VÀ TRIỂN KHAI HỆ THỐNG} (Trích xuất
từ tài liệu README.md và hướng dẫn vận hành)

\textbf{1. Yêu cầu phần mềm (Prerequisites)} Để chạy được mã nguồn dự
án, máy tính cần cài đặt sẵn các công cụ sau:

\begin{itemize}
\item
  Node.js: Phiên bản 18.x LTS trở lên.
\item
  Git: Để quản lý mã nguồn.
\item
  Visual Studio Code: Trình soạn thảo mã nguồn khuyến nghị.
\end{itemize}

\textbf{2. Cấu hình Biến môi trường (.env)} Tạo file .env trong thư mục
Backend với các thông số cấu hình sau:

\# -\/-\/- CẤU HÌNH SERVER -\/-\/-

PORT=3000

NODE\_ENV="development"

\# -\/-\/- CẤU HÌNH CƠ SỞ DỮ LIỆU (SQLITE) -\/-\/-

\# Đường dẫn đến file database cục bộ

DATABASE\_URL="file:./dev.db"

\# -\/-\/- CẤU HÌNH AI \& BẢO MẬT -\/-\/-

\# API Key của Google Gemini (Lấy tại aistudio.google.com)

GEMINI\_API\_KEY="AIzaSyA\_YOUR\_SECRET\_KEY\_HERE"

\# Secret Key dùng để mã hóa Token đăng nhập Admin

JWT\_SECRET="danatravel\_super\_secret\_key\_2024"

\textbf{3. Các bước cài đặt chi tiết}

Bước 1: Khởi chạy Backend (Máy chủ)

\begin{quote}
cd Backend

npm install ~ ~ ~ ~ ~ ~ ~ ~ \# 1. Cài đặt các thư viện phụ thuộc
(node\_modules)

npx prisma migrate dev ~ ~ ~\# 2. Khởi tạo cấu trúc Database từ file
schema

npx prisma db seed ~ ~ ~ ~ ~\# 3. Nạp dữ liệu mẫu (Locations, Knowledge)

npm run dev ~ ~ ~ ~ ~ ~ ~ ~ \# 4. Chạy server ở chế độ Development
\end{quote}

Kết quả: Server sẽ lắng nghe tại \url{http://localhost:3000}.

Bước 2: Khởi chạy Frontend (Giao diện)

cd Frontend

npm install ~ ~ ~ ~ ~ ~ ~ ~ \# 1. Cài đặt thư viện React, Vite, Tailwind

npm run dev ~ ~ ~ ~ ~ ~ ~ ~ \# 2. Khởi động Web Server

Kết quả: Truy cập ứng dụng tại \url{http://localhost:5173}.

\textbf{PHỤ LỤC 2. THIẾT KẾ CƠ SỞ DỮ LIỆU (DATABASE SCHEMA)} (Trích xuất
từ file Backend/prisma/schema.prisma)

Dưới đây là mã nguồn định nghĩa các bảng dữ liệu chính bằng ngôn ngữ
Prisma Schema:

// 1. Bảng Địa điểm - Lưu trữ thông tin cốt lõi

model Location \{

~ id ~ ~ ~ ~ ~String ~ @id @default(uuid())

~ name ~ ~ ~ ~String

~ type ~ ~ ~ ~String ~ // Phân loại:
\textquotesingle visit\textquotesingle{} (tham quan),
\textquotesingle food\textquotesingle{} (ăn uống)

~ address ~ ~ String?

~ lat ~ ~ ~ ~ Float? ~ // Tọa độ Vĩ độ

~ lng ~ ~ ~ ~ Float? ~ // Tọa độ Kinh độ

~ ticket ~ ~ ~Float? ~ // Giá vé / Chi phí trung bình

~ openTime ~ ~String? ~// Giờ mở cửa (Format: "HH:mm")

~ closeTime ~ String? ~// Giờ đóng cửa

~

~ // Các trường JSON linh hoạt

~ tags ~ ~ ~ ~String? ~// Lưu mảng sở thích: {[}"biển", "sống ảo"{]}

~ menu ~ ~ ~ ~String? ~// Lưu danh sách món ăn: {[}\{"name": "Mì Quảng",
"price": 30000\}{]}

~

~ description String?

~ suggestedDuration Int? // Thời gian tham quan gợi ý (phút)

~ createdAt ~ DateTime @default(now())

~ updatedAt ~ DateTime @updatedAt

\}

// 2. Bảng Tri thức - Dùng cho Chatbot RAG

model Knowledge \{

~ id ~ ~ ~ ~String ~ @id @default(uuid())

~ question ~String ~ // Câu hỏi mẫu

~ answer ~ ~String ~ // Câu trả lời chuẩn

~ keywords ~String? ~// Từ khóa phục vụ tìm kiếm Full-text

~ updatedAt DateTime @updatedAt

\}

// 3. Bảng Xu hướng - Phục vụ Thống kê Admin

model SearchTrend \{

~ id ~ ~ ~ ~String ~ @id @default(uuid())

~ tags ~ ~ ~String ~ // Các tag người dùng đã chọn

~ duration ~String? ~// "3 ngày", "4 ngày"

~ budget ~ ~Float? ~ // Ngân sách người dùng nhập

~ people ~ ~Int? ~ ~ // Số lượng người đi

~ createdAt DateTime @default(now())

\}

\textbf{PHỤ LỤC 3. MÃ NGUỒN THUẬT TOÁN LẬP LỊCH TRÌNH (CORE ALGORITHM)}
(Trích xuất từ file Backend/src/utils/generate-day-schedule-strict.js)

Đây là đoạn mã thể hiện tư duy giải thuật Tham lam (Greedy) kết hợp Quay
lui (Backtracking) để chọn địa điểm tối ưu:

/**

~* Hàm lõi: Tìm địa điểm tốt nhất tiếp theo

~* @param \{Object\} currentLocation - Tọa độ điểm hiện tại

~* @param \{Array\} candidates - Danh sách địa điểm ứng viên

~* @param \{Number\} remainingBudget - Ngân sách còn lại

~* @param \{String\} currentTime - Thời gian hiện tại (VD: "09:00")

~*/

function findBestNextSpot(currentLocation, candidates, remainingBudget,
currentTime) \{

~ ~ let bestSpot = null;

~ ~ let minScore = Infinity; // Điểm số càng thấp càng tốt (Ưu tiên gần
+ rẻ)

~ ~ for (const spot of candidates) \{

~ ~ ~ ~ // 1. Tính khoảng cách địa lý (Haversine Formula)

~ ~ ~ ~ const distance = calculateDistance(

~ ~ ~ ~ ~ ~ currentLocation.lat, currentLocation.lng,

~ ~ ~ ~ ~ ~ spot.lat, spot.lng

~ ~ ~ ~ );

~ ~ ~ ~ // 2. Kiểm tra các ràng buộc cứng (Hard Constraints)

~ ~ ~ ~ // - Phải đủ tiền vé

~ ~ ~ ~ if (spot.ticket \textgreater{} remainingBudget) continue;

~ ~ ~ ~ // - Phải đang mở cửa

~ ~ ~ ~ if (!isOpenAt(spot.openTime, spot.closeTime, currentTime))
continue;

~ ~ ~ ~ // 3. Tính điểm heuristic (Trọng số: 70\% khoảng cách, 30\% giá
vé)

~ ~ ~ ~ // Mục tiêu: Tìm điểm gần nhất nhưng không quá đắt

~ ~ ~ ~ const score = (distance * 0.7) + (spot.ticket * 0.0001 * 0.3);

~ ~ ~ ~ if (score \textless{} minScore) \{

~ ~ ~ ~ ~ ~ minScore = score;

~ ~ ~ ~ ~ ~ bestSpot = \{ ...spot, estimatedDistance: distance \};

~ ~ ~ ~ \}

~ ~ \}

~ ~ return bestSpot;

\}

/**

~* Hàm chính: Sinh lịch trình cho 1 ngày

~*/

export const generateDaySchedule = async (dayIndex, budget, tags)
=\textgreater{} \{

~ ~ // ... Khởi tạo các biến ...

~ ~

~ ~ // Chiến lược: Cố định bữa ăn trước (Skeleton Strategy)

~ ~ const lunch = pickRandomFoodSpot(budget * 0.2, tags);

~ ~ const dinner = pickRandomFoodSpot(budget * 0.2, tags);

~ ~ // Điền vào Slot Sáng (Từ Khách sạn -\textgreater{} Ăn trưa)

~ ~ let currentPos = HOTEL\_LOCATION;

~ ~ const morningSpot = findBestNextSpot(currentPos, attractions,
budget, \textquotesingle08:00\textquotesingle);

~ ~

~ ~ // ... Logic lặp để điền các slot còn lại ...

~ ~ return dailyItinerary;

\};

\textbf{PHỤ LỤC 4. MÃ NGUỒN TÍCH HỢP CHATBOT AI (RAG SERVICE)} (Trích
xuất từ file Backend/src/services/chatbot.service.js)

Đoạn mã xử lý logic Retrieval-Augmented Generation (RAG) để đảm bảo
Chatbot trả lời chính xác dữ liệu từ Database:

import \{ GoogleGenerativeAI \} from "@google/generative-ai";

import \{ prisma \} from "../utils/prisma.js";

const genAI = new GoogleGenerativeAI(process.env.GEMINI\_API\_KEY);

const model = genAI.getGenerativeModel(\{ model: "gemini-1.5-flash" \});

export const processUserMessage = async (userMessage) =\textgreater{} \{

~ try \{

~ ~ // BƯỚC 1: TRUY XUẤT DỮ LIỆU (RETRIEVAL)

~ ~ // Tìm kiếm từ khóa trong bảng Location và Knowledge

~ ~ const relevantLocations = await prisma.location.findMany(\{

~ ~ ~ where: \{

~ ~ ~ ~ OR: {[}

~ ~ ~ ~ ~ ~ \{ name: \{ contains: userMessage \} \},

~ ~ ~ ~ ~ ~ \{ tags: \{ contains: userMessage \} \}

~ ~ ~ ~ {]}

~ ~ ~ \},

~ ~ ~ take: 3 // Lấy 3 kết quả tốt nhất

~ ~ \});

~ ~ // BƯỚC 2: TẠO NGỮ CẢNH (CONTEXT AUGMENTATION)

~ ~ let contextData = "Dữ liệu hệ thống:\textbackslash n";

~ ~ relevantLocations.forEach(loc =\textgreater{} \{

~ ~ ~ ~ contextData += `- \$\{loc.name\}: Giá vé \$\{loc.ticket\} VNĐ,
Địa chỉ: \$\{loc.address\}\textbackslash n`;

~ ~ \});

~ ~ // BƯỚC 3: XÂY DỰNG PROMPT

~ ~ const prompt = `

~ ~ ~ Bạn là trợ lý ảo DanaTravel.

~ ~ ~ Dựa vào dữ liệu chính xác sau đây:

~ ~ ~ \$\{contextData\}

~ ~ ~

~ ~ ~ Hãy trả lời câu hỏi của người dùng: "\$\{userMessage\}"

~ ~ ~ Yêu cầu: Trả lời ngắn gọn, thân thiện, chỉ dùng thông tin được
cung cấp.

~ ~ `;

~ ~ // BƯỚC 4: GỌI GOOGLE GEMINI (GENERATION)

~ ~ const result = await model.generateContent(prompt);

~ ~ const response = result.response.text();

~ ~ return response;

~ \} catch (error) \{

~ ~ console.error("Chatbot Error:", error);

~ ~ return "Xin lỗi, tôi đang gặp sự cố kết nối. Vui lòng thử lại sau.";

~ \}

\};

\end{document}
